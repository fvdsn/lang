\section{Traduction du programme en code interne}
\subsection{Création de l'arbre syntaxique du programme}
L'arbre syntaxique est consitué de \emph{Term} afin
qu'il puisse être généré directement par l'analyseur syntaxique. 
La méthode getChildList() de la classe Term renvoie la liste des \emph{Term} enfants.


Le code de l'analyser syntaxique prend en entrée un flux Term qui sont tous terminaux. 
L'analyseur syntaxique n'a besoin que des méthodes définie dans l'interface Term pour construire l'arbre.
Voici cette interface : 
{\tiny
\begin{verbatim}
 

public interface Term {
	  /**
	  * 
	  * @return true if the term is a terminal term
	  */
	  public boolean isTerminal();
	  
	  /**
	  * 
	  * @return Le type du term pour l'analyseur syntaxique.
	  */
	  public String getType();
	
	  /**
	  * 
	  * @return la valeur du term, c'est à dire la chaine de caractère
	  * parsée par l'analyseur lexical.
	  * Si le term n'est pas terminal, il n'a pas de valeur
	  */
	  public String getValue();
	
	  /**
	  * Modifie la valeur du term
	  * @param v la nouvelle valeur du term
	  * @return this
	  */
	  public Term setValue(String v);
	
	  /**
	  * Renvoie la liste des enfants du terme, cette liste n'est pas vide que
	  * si le term n'es pas terminal
	  * @return
	  */
	  public List<Term> getChildList();
	
	  /**
	  * Imprime l'arbre qui est représenté par le term
	  * Indent définit le niveau d'indentation, lorsqu'on imprime
	  * l'arbre entier le point de départ est 0
	  * @param indent
	  */
	  void	printTree(int indent);
	
	
}
\end{verbatim}}


Une fois le code déjà donné dans le chapitre 6 exécuté, l'arbre (le Term) passe dans le treeOrganiser.

La première chose à faire avec l'arbre généré par l'arbre brut est
de mettre tout ce qui est au même niveau de parenthèses au même niveau
dans l'arbre. Pour cela on le parcourt et on fusionne récursivement tous
les $l\_blocks$ et les $l\_lists$ dans leurs parents. Comme les \emph{Term}
gardent en mémoire le type donné à l'analyse syntaxique, les terminaux
récupèrent automatiquement le bon type. 
 
\begin{figure}
 \centering
 \includegraphics[bb=0 0 428 264]{./restruc.png}
 % restruc.png: 428x264 pixel, 72dpi, 15.10x9.31 cm, bb=0 0 428 264
 \caption{Schema simplifié de la restructuration de l'arbre}
 \label{restruc}
\end{figure}


Voici la spécification de la seul méthode public du \textit{TreeOrganiser} :
\begin{verbatim}
 /**
  * Contracte l'arbre sortit par l'analyseur syntaxique. 
    Tout les l_list et l_block sont
  * retiré, à la fin il ne reste que des l0 qui sont 
    soit des terminaux ou des listes de l0.
  * @return L'arbre représenté sous la forme d'un lexicalTerm
  */
public LexicalTerm contract()
\end{verbatim}

La structure de donnée LexicalTerm est très similaire à Term, elle implémente l'interface on a rajouté une méthode getLexicalTerm
qui est le type donnée par le parser lexical. Ces types ont été listé dans la partie sur l'analyseur lexical. Une fois 
l'arbre mis en forme et les types des termes réels révélés (pas ceux juste présent pour construire l'arbre), on passe à la traduction.

\section{traduction}
Il y a trois grande famille de traduction, la traduction des définitions de méthode, la traduction des instructions et la traduction des conditions.
La traduction des définitions de méthode est assez facile car on sait qu'un programme en Happy est une liste de méthode. Donc les enfants
de la racine sont les méthodes, l'enfant 0 de l'enfant est le mot clé, l'enfant 1 sont les paramètres et l'enfants 2 est la suite d'instruction.
\begin{verbatim}

/**
	 * Point de départ de l'exploration de l'arbre. 
	 * t est un term qui une liste de term qui est en accord avec la définition
	 * des méthodes ou des fonctions.
	 * Si t n'est pas conforme aux définitions, une erreur de syntaxe est 
  lancée et le programme
	 * se termine
	 * @param t 
	 * @return la Cmethod 
	 */
	private Cmethod analyseMethod(LexicalTerm t) { 

 
\end{verbatim}

Une fois les arguments et le nom de la méthode récupéré il faut commencé à ajouté toutes les instructions à au corps de la méthode.
La méthode privée addBody crée une liste vide qui accueillera toutes les instructions de la méthode. Cette table permet de placer en premier
les instructions qui se trouvent au fond de l'arbre pour gérer correctement les appels imbriqués.

Les instructions dans le langage Happy sont toutes des expressions, c'est-à-dire qu'elle renvoie une valeur. Contrètement dans la traduction
cela ce fait par l'assignation d'une valeur (qui dépend du type d'instruction) variable annonyme qui sera renvoyé à la méthode appelante. Cette méthode
appelante utilisateur la variable si elle en a besoin.

Voici le code de la méthode \textit{addOutput}
{\tiny
\begin{verbatim}

 
/**
 * Ajoute l'instruction output à la liste des instructions, les expréssions
 * imbriquée dans cette instruction seront ajouter avant l'exécution de celle-ci
 * @param child la liste des arguments de l'instruction write, 
   size == 2, l'élément 0 est le term write
 * et le seconde une expression
 * @param table la table des variables local
 * @param body la table contenant les instructions de la méthode 
 * @param counter le compteur des variables annonymes
 * @return la variable qui contient la valeur de retour de l'instruction
 */
private Cvar addOutput(List<LexicalTerm> child, MethodTable table,
   List<Ccmd> body, AnonymousCounter counter) {
  //si l'expression est terminal on peut la traiter tout de suite
  if(child.get(1).isTerminal()) {
    Cvar v1 = null;
    if(WordIdentifier.isRexpr(child.get(1))) {
      Crexpr r = WordIdentifier.getRexpr(child.get(1)); 
      String name = counter.getAnonymousName(); 
      table.addLocal(name);
      v1 = new Cvar(name);				
      body.add(new Cass(v1,r ));
    }
    else if(child.get(1).getLexicalTerm().equals("id")) {
      v1 = new Cvar(child.get(1).getValue());
      table.addLocal(child.get(1).getValue());
    }
    //erreur terminal mais ni une expression ni un id
    else {
      System.out.println("Expected number id or expr");
      System.out.println("in write");
      System.exit(15);
    }
    Clexpr[] expr = {v1};
    Coutput out = new Coutput(expr);
    body.add(out);
    return v1;
  }
  //sinon on explore la commande et on récupère la valeur de l'expression
  else {
    Cvar v1 = exploreCmd(child.get(1), table, body, counter);
    Clexpr[] expr = {v1};
    Coutput out = new Coutput(expr);
    body.add(out);
    return v1;
  }
}
 \end{verbatim}
}
Dans ce code on voit bien les deux cas de figure : les arguments de l'instruction sont des terminaux ou non. 
On voit aussi l'utilisation des variables annonymes grâce à la classe AnonymousCounter qui a comme seul objectif de fournir des noms différents
pour chaque variable annonyme d'une méthode. 

 
Voici un petit de l'exécution de la méthode addOutput :

Considérons le code Happy suivant :
\begin{verbatim}
(write 
  (set a 
    (+ 7 8)
  )
)
sera traduit en:
A_0 = 7 ; //l'expression 7 est assignée à A_0
A_1 = 8 ; //idem pour 8
A_2 = (A_0 + A_1) ; //l'expression (+ A_0 A_1) est assignée à A_2
a = A_2 ; //assignation de l'expression (+ A_0 A_1) à a en passant par A_2
A_3 = a ; //set renvoie la variable qu'il vient d'assigner
write(A_3) ;

\end{verbatim}

Les conditions fonctionne un peu comme l'exploration des commandes, mais ici il faut distinguer deux types de conditions.
Les conditions avec un opérateur logique (and, or, not) et les conditions avex un opérateur de relation. Les première ne peuvent avoir 
comme argument que des conditons tandis que les second ne peuvent avoir comme argument que des expressions. 

Voici un exemple de condition avec un opérateur logique :
{\tiny
\begin{verbatim}


/**
 * Cette fonction renvoie une condition de type and.
 * @param child la liste des Term qui forme la conditions, 
    size == 3 et les éléments 1 et 2 doivent être des conditions
 * @param table la table des variables locales
 * @param body la table contenant la liste des commandes du corps de la méthode
 * @param counter le compteur de variable annonyme
 * @return La condition and.
 */
private Ccond getAnd(List<LexicalTerm> child, MethodTable table, 
  List<Ccmd> body, AnonymousCounter counter) {
  if(child.get(1).isTerminal() || child.get(2).isTerminal()) {
    System.out.println("Syntax error at AND, expected condition");
    System.exit(16);
  }
  return new Cand(getCond(child.get(1), table, body, counter),
      getCond(child.get(2), table, body, counter));
}
 
\end{verbatim}}

Ici encore on voit les appels récursif à getCond (en effet l'appel de getAnd vient uniquement d'un appel à getCond). 
Les conditons avec un opérateur de relation fonctionne exactement comme les instructions. 
L'implémentation avec les expressions imbriquées dans les relations pose un petit soucis. Les expressions sont placé au dessus du if
ou du while et leur résultat sont placés dans une variable annonyme. Cette variable est ensuite utilisée dans la condtion.
Dans un if ceci ne pose pas de problème mais dans un while on risque d'avoir très vite une boucle infinie car les expressions ne
sont plus réévaluée une fois dans le while. (Il faudrait donc aussi rajouté les instructions imbriquée à la fin du while mais l'implémentation
dans sa forme actuelle ne permet pas de le faire facilement). C'est pourquoi le code suivant est vivement déconseillé
{\tiny
\begin{verbatim}
[code qui écris les chiffre de 1 à 10]
 (
        (fun (main) (
                (set i 0)
                (while (<= (set i (+ i 1)) 10) (
                        (write i)

                ))

        ))
)
[La sortie est tout autre => une infinité de 1]

[Voici le code conseillé]
 (
        (fun (main) (
                (set i 1)
                (while (<= i 10) (
                        (write i)
                        (set i (+ i 1))

                ))

        ))
)

[Ce code va effectivement sortir les chiffres de 1 à 10]
\end{verbatim}}

\section{traduction du code en code interne}
Ici nous avons utilisé la méthode Cprog.translate(). Qui nous à poser quelque problème lorsque qu'il y avait dans la traduction en code 
structuré. En effet les erreurs renvoyée ne sont pas des plus claires.

Voici le premier programme de la démo qui fait la somme de 1 à n.
{\tiny
\begin{verbatim}
 


(

[Commentaire]
(fun (main) (
  (set i 0)
  (while (<= i 10)
  (
    (write (sum i))
    (set i (+ i 1))
  ))
))

(fun (sum n) (
  (if (= n 0)
    (return 0)
  )
  (return (+ n (sum (- n 1))))
  ))
)


Traduction en code structuré

main()
{
  i = 0 ;
  A_0 = i ;
  A_1 = 10 ;
  while (i <= A_1)
    {
      A_2 = sum(i) ;
      write(A_2) ;
      A_3 = 1 ;
      A_4 = (i + A_3) ;
      i = A_4 ;
      A_5 = i ;
      A_6 = 0 ;
    }
  A_7 = 0 ;
  A_8 = 0 ;
}


sum(n)
{
  A_0 = 0 ;
  if (n == A_0)
    {
      A_1 = 0 ;
      return (A_1) ;
    }
  A_2 = 0 ;
  A_3 = 1 ;
  A_4 = (n - A_3) ;
  A_5 = sum(A_4) ;
  A_6 = (n + A_5) ;
  return (A_6) ;
  A_7 = 0 ;
}

Traduction en code interne


method main()
{ lab16
  [ lab16 : i#1 := 0 ; go to lab15]
  [ lab15 : A_0#2 := i#1 ; go to lab14]
  [ lab14 : A_1#3 := 10 ; go to lab5]
  [ lab5 : if A_1#3 < i#1 then go to lab3 else go to lab4]
  [ lab3 : A_7#9 := 0 ; go to lab2]
  [ lab2 : A_8#10 := 0 ; go to lab1]
  [ lab4 : skip ; go to lab13]
  [ lab13 : #11 := i#1 ; go to lab12]
  [ lab12 : A_2#4 := sum(#11) ; go to lab11]
  [ lab11 : write(A_2#4) ; go to lab10]
  [ lab10 : A_3#5 := 1 ; go to lab9]
  [ lab9 : A_4#6 := i#1 + A_3#5 ; go to lab8]
  [ lab8 : i#1 := A_4#6 ; go to lab7]
  [ lab7 : A_5#7 := i#1 ; go to lab6]
  [ lab6 : A_6#8 := 0 ; go to lab5]
} lab1 result#0
end of method main.

method sum(n)
{ lab29
  [ lab29 : A_0#2 := 0 ; go to lab28]
  [ lab28 : if n#1 == A_0#2 then go to lab27 else go to lab25]
  [ lab27 : A_1#3 := 0 ; go to lab26]
  [ lab26 : result#0 := A_1#3 ; go to lab17]
  [ lab25 : A_2#4 := 0 ; go to lab24]
  [ lab24 : A_3#5 := 1 ; go to lab23]
  [ lab23 : A_4#6 := n#1 - A_3#5 ; go to lab22]
  [ lab22 : #10 := A_4#6 ; go to lab21]
  [ lab21 : A_5#7 := sum(#10) ; go to lab20]
  [ lab20 : A_6#8 := n#1 + A_5#7 ; go to lab19]
  [ lab19 : result#0 := A_6#8 ; go to lab17]
} lab17 result#0
end of method sum.

Et finalement l'exécution
0
1
3
6
10
15
21
28
36
45
55


\end{verbatim}}