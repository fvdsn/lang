\chapter{Sémantique Opérationelle}


\section{Domaines sémantiques}
\paragraph{Valeurs} Val $\doteq$ Entier + Référence + \{null\} + \{undefined\} + \{Error\}

Une Valeur représente soit un entier soit une réference vers une autre valeur
ou un autre objet. Les cas limite sont aussi compris. Il s'agit de Null pour 
une réference ne pointant vers rien  et de NonInit pour un Int qui n'aurait 
pas été initialisé. \emph{Undefined} représente une variable dont la valeur n'a
pas encore été spécifiée et dont le type n'est donc pas connu. Les champs
des objets fraichement alloués sont de type \emph{undefined}.

\paragraph{Environnement} Env $\doteq \mathbb{X}$ + \{this\} $\rightarrow$ Val

Un environnement est une fonction assurant la correspondence entre une variable ou \emph{this} et la valeur de celle-ci.

\paragraph{Store} Store $\doteq$ Ref $\rightarrow$ $<n,<v_1,v_2,...,v_n> >$	%TODO la flèche comme ca : -|-> 

Store est une fonction. La valeur d'une variable pointant vers un objet est une référence. Plusieurs 
variables peuvent donc avoir des références de même valeur et pointer vers le 
même objet. Le store est une fonction qui permet d'établir une correspondance 
entre la référence et la valeur de l'objet.

\paragraph{Pile} Pile $\doteq$ $<$ Env, Label, X, $<$ Pile $> >$ 

La pile permet au programme de gérer les changements d'environnement qui interviennent
à chaque appel ou retour de fonction. La pile est une pile contenant à chaque étage 
l'environnement courant, le label et la variable de retour. Une pile n'est bien évidemment pas infinie,il est cependant assez difficile de modéliser cette contrainte de manière formelle.

\paragraph{Etats} Etat $\doteq <e,l,s,P,in,out>$\\

$e \in $ Env   \\
$l \in $ Label \\
$s \in $ Store \\
$P \in $ Pile  \\
L'état comprend tout ce qui est nécessaire pour continuer l'exécution du programme, en excluant le
code source de celui-ci, cela veut dire l'environnement, le label de l'instruction courante, le store,
ainsi que la pile d'appel de fonction. 

\paragraph{in} In $\doteq$ Pile $\rightarrow$ Val

La fonction In permet de lire la première valeur de la pile ( donc la dernière valeur entrée par l'utilisateur ).

\paragraph{Out} $\doteq$ Val  $\rightarrow$ Val

La fonction Out permet d'écrire Val au sommet de la pile, Elle renvoie par ailleurs cette même valeur.

\section{Philosophie et choix de design}
	Définir la sémantique opérationnelle ne se limite pas à formaliser des concepts génériques
	comme présenté dans le précédent chapitre. Il faut souvent choisir entre plusieurs alternatives
	et ces choix étendent ou limitent les possibilités du langage.

	Afin de garantir une certaine cohérence dans tout cela, nous avons choisi de tenir une 
	certaine ligne de conduite justifiée ici.

	Tout d'abord deux observations: premièrement le langage \emph{SLIP} n'est pas destiné à 
	être utilisé directement par le programmeur puisque nous allons définir une syntaxe plus 
	agréable dans une seconde partie du projet. Ensuite, dans tout langage il faut choisir 
	un équlibre entre flexibilité et facilité de détection d'erreurs. Par exemple un programme
	typé dynamiquement est plus rapide à écrire mais son exécution est plus difficile à prévoir,
	et donc la garantie que le programme est correct est plus difficile à fournir.

	Nous avons fait le choix d'avoir un langage qui n'essaie pas de deviner ce que le programmeur
	voulait faire, et reporte les erreurs dès que possible. 

	Ainsi, c'est au programmeur de s'assurer qu'il n'y a pas de divisions par zéro. Il nous semble
	plus important que le programme s'arrête en indiquant l'erreur, plutôt que de poursuivre en
	faisant des approximations, en ne prévenant pas le programmeur de la non exactitude des résultats
	obtenus. 

\section{Fonctions sémantiques}
\paragraph{Conditions} B: Cond $\rightarrow$ Env $\rightarrow$ Store $\rightarrow \{true, false, error\}$

$B[$ expr1 cop expr2 $]$ e s\\
$v_1 = V[$expr1$]$ e s\\
$v_2 = V[$expr2$]$ e s\\
$\xi[$cop$]$ $v_1$ $v_2 = \xi_1$ $v_1$ $v_2$ si cop$= '<'$\\
$\xi[$cop$]$ $v_1$ $v_2 = \xi_2$ $v_1$ $v_2$ si cop$= '='$\\
\\
$\xi_1 :$ Val $\rightarrow$ Val $\rightarrow$ $\{true,false,error\}$\\
$\xi_1$ $v_1$ $v_2 = $ true si $v_1 < v_2$ avec $v_1,v_2 \in $ Int\\
$\xi_1$ $v_1$ $v_2 = $ false si $v_1 \geq v_2$ avec $v_1,v_2 \in $ Int\\
$\xi_1$ $v_1$ $v_2 = $ error sinon\\

$\xi_2 :$ Val $\rightarrow$ Val $\rightarrow$ $\{true,false,error\}$\\
$\xi_2$ $v_1$ $v_2$ =  true si $v_1 = v_2$ avec $v_1,v_2 \in $ Int\\
$\xi_2$ $v_1$ $v_2$ =  true si $v_1 = v_2$ avec $v_1,v_2 \in $ Ref\\
$\xi_2$ $v_1$ $v_2$ =  true si $v_1,v_2 \in $ \{null\} \\
$\xi_2$ $v_1$ $v_2$ =  false si $v_1 \neq v_2$ avec $v_1,v_2 \in $ Int\\
$\xi_2$ $v_1$ $v_2$ =  false si $v_1 \neq v_2$ avec $v_1,v_2 \in $ Ref\\
$\xi_2$ $v_1$ $v_2$ =  false si $v_1  \in $ \{null\} et $v_2 \in Ref$\\
$\xi_2$ $v_1$ $v_2$ =  false si $v_1  \in $ Ref et $v_2 \in $ \{null\}\\
$\xi_2$ $v_1$ $v_2$ =  error sinon\\

Cette fonction définit le comportement des conditions. Nous avons choisi une
définition assez stricte ou seul des élèments de même type peuvent être comparés. 
Les comparaisons avec \emph{undefined} renvoie une erreur car on ne sait pas se 
prononcer sur la comparaison d'élément inconnus. 

\paragraph{Designateur} $D$: Des $\rightarrow$ Env $\rightarrow$ Store $\rightarrow$ Val + {Ref,Int} +\{$error$\} \\
$D[x] = x$ si $x \in \mathbb{X}$ \\
$D[x]$ = $error$ sinon \\
$D[x.i] = [ $Env$(x), i ]$ si $ x \in \mathbb{X}$ et $x \in dom($Env$) $ et Env($x$) $\in dom($Store$)$ et si
	$<n,<v_1,v_2,...,v_n> > =$Store($x$) et $ i \in [0,n[$ \\
$D[x.i]$ = $error$ sinon \\
$D[this.i]$ = \{Env$(this), i\}$ si Env($this$) $\in dom($store$)$ et si
	$< n,< v ...>> = $Store($this$) et $ i \in [0,n[$ \\
$D[this.i]$ = $error$ sinon \\
Un désignateur renvoie la variable correspondante, ou une paire référence, indice, selon la forme du désignateur.
le designateur renvoie une erreur si l'on accède à un champ d'un objet non défini, ou que l'indice est en dehors
des limites de l'objet. 

\paragraph{Expressions} $V:$ Expr $\rightarrow$ Env $\rightarrow$ Store $\rightarrow$ Val + \{error\}\\
$V[Val] = Val$\\
$V[Des] = Env[D[$Des$]]$ si $D[$Des$] \in dom($Env$)$ \\
$V[Des] = n $ si $D[$Des$]$ = $\{ref,i\}$ et $i = 0$ et Store($ref$) = $<n,<...>>$ \\
$V[Des] = v_i$ si $D[$Des$]$ = $\{ref,i\}$ et $i \neq 0$ et Store($ref$) = $<n,<v_1,...,v_i,...,v_{n}>$\\
$V[Des] = error $ sinon\\
\\
$V[$Expr1 op Expr2$] = V_+[E_1 E_2] $ si op $= '+'$\\
$V[$Expr1 op Expr2$] = V_-[E_1 E_2] $ si op $= '-'$\\
$V[$Expr1 op Expr2$] = V_*[E_1 E_2] $ si op $= '*'$\\
$V[$Expr1 op Expr2$] = V_/[E_1 E_2] $ si op $= '/'$\\
$V[$Expr1 op Expr2$] = V_\%[E_1 E_2] $ si op $= '\%'$\\
$V[$Expr1 op Expr2$] = error$ sinon\\
\\
$V_+[E_1 E_2] = V[E_1] + V[E_2]$ si $V[E_1]$ et $V[E_2] \in$ int \\
$V_+[E_1 E_2] = error$ sinon.\\
\\
$V_-[E_1 E_2] = V[E_1] - V[E_2]$ si $V[E_1]$ et $V[E_2] \in$ int \\
$V_-[E_1 E_2] = error$ sinon.\\
\\
$V_*[E_1 E_2] = V[E_1] * V[E_2]$ si $V[E_1]$ et $V[E_2] \in$ int \\
$V_*[E_1 E_2] = error$ sinon.\\
\\
$V_/[E_1 E_2] = V[E_1] / V[E_2]$ si $V[E_1]$ et $V[E_2] \in$ int, et que  $V[E_2] \neq 0$\\
$V_/[E_1 E_2] = error$ sinon.\\
\\
$V_\%[E_1 E_2] = V[E_1] \% V[E_2]$ si $V[E_1]$ et $V[E_2] \in$ int et que $V[E_2] \neq 0$\\
$V_\%[E_1 E_2] = error$ sinon.\\
\\

Il est bon de noter qu'aucun dépassement des bornes n'est possible, et ce car l'ensemble des valeurs comprend tout l'ensemble des naturels (Cela sera implémenté en java grâce à la classe BigInteger, mais il n'est pas encore temps d'y penser ;-) )

\paragraph{Affectations} $A :$ Ass $\rightarrow$  Env $\rightarrow$ Store $\rightarrow$ {Env,Store} + \{$error$\}  \\
$A[[des := expr]] e s = A1[[des]][[expr]] e s$\\
$A[[x := new/i ]] e s = A2[[x]][[i]] e s$\\
Sinon $error$
\newline
$A1 : $Des$ \rightarrow $Expr$ \rightarrow $Env$ \rightarrow $Store$ \rightarrow$ {NewEnv,NewStore} + \{$error$\} \\ 
Si $Env[D[$Des$]] \in {undefined}$ et $V[$Expr$] \neq error$ alors NewEnv = Env $\bigoplus$ $D[$Des$]$ $\rightarrow$ $V[$Exp$]$ et NewStore = Store\\
\newline
Si $Env[D[$Des$]] \in$ Int et $V[$Expr$] \in$ Int alors NewEnv = Env $\bigoplus$ $D[$Des$]$ $\rightarrow$ $V[$Exp$]$ et NewStore = Store\\
\newline
Si $D[$Des$] = \{ref, i\} \in$ \{Ref, Int\} et $<n,<...,v_i,...> = Store[ref]$ et $i \neq 0$ et $v_i \in {undef}$ et $V[$Expr$] \neq error$ alors NewEnv = Env et NewStore = Store $\bigoplus ref \rightarrow <n,<...v_{i-1},V[$Expr$],v_{i+1}...>>$\\ 
\newline
Si $D[$Des$] = \{ref, i\} \in$ \{Ref, Int\} et $<n,<...,v_i,...> = Store[ref]$ et $i \neq 0$ et $v_i \in$Int et $V[$Expr$] \in $Int alors NewEnv = Env et NewStore = Store $\bigoplus ref \rightarrow <n,<...v_{i-1},V[$Expr$],v_{i+1}...>>$\\ 
\newline
Si $D[$Des$] = \{ref, i\} \in$ \{Ref, Int\} et $<n,<...,v_i,...> = Store[ref]$ et $i \neq 0$ et $v_i \in$Ref + \{$null$\} et $V[$Expr$] \in $Ref + \{$null$\} alors NewEnv = Env et NewStore = Store $\bigoplus ref \rightarrow <n,<...v_{i-1},V[$Expr$],v_{i+1}...>>$\\ 
\newline
$A2 : $X$ \rightarrow $Int$ \rightarrow $Env$ \rightarrow $Store$ \rightarrow$ {NewEnv,NewStore} + \{$error$\} \\ 
Si $Env[$x$] \in Ref+{undefined}$ et $i > 0$ et $r \in Ref$ et $r \not\in dom(Store)$  alors  
	NewEnv = Env $\bigoplus x \rightarrow r$ et NewStore = Store $\bigoplus r \rightarrow <i,<v_0,...,v_j,...,v_{i}>> | \forall(j) v_j = undefined$\\
	$error$ sinon\\

L'opérateur $\bigoplus$ désignant la surimpression fonctionnelle, c'est à dire que  pour une fonction $f : x \rightarrow y $ , la surimpression par un couple  $ a \rightarrow b $, signifie que $f(x)=y$ pour $x \neq a$ et $f(x)=b$ pour $x = a$\\

L'affectation force le respect des types. Il n'est pas possible d'assigner une référence à une variable qui était 
précédement un entier, et inversément. Comme il n'y a pas d'instruction permettant de savoir quel est le type d'une variable il 
nous a semblé plus raisonnable d'imposer le respect des types afin que le programmeur puisse savoir à tout moment
de quel type est chaque variable. 

Il n'est pas non plus possible de modifier la taille d'un objet, et lorsque ceux-ci sont instanciés, tous leur champs sauf
celui à zéro sont mis à \emph{undefined}.  

\paragraph{In/Out}
L'input est représenté par une pile de valeurs entières sur laquelle le programme lit ses entrées. 
In = $< val \in $Int$, <$In$>>$\\
$I : in \rightarrow Env \rightarrow In \rightarrow {In,Env} + {error}$\\
Si $in = read x$ et $x \in X$ et $ Env(x) \in Int + {undefined} $ et $In = < val, <In2>>$ alors $NewIn = In2$ et $NewEnv = Env \bigoplus x \rightarrow val $\\
$error$ sinon\\

De manière similaire l'output est représenté par une pile de valeur entière sur laquelle le programme lit ses entrées. 
Out = $< val \in $char*$ <$Out$>>$\\
$O : out \rightarrow Env \rightarrow  Out \rightarrow NewOut $\\
Si $out = write x$ et $x \in dom(Env)$ et $Env(x) \in Int$ NewOut = $<tochar(Env(x)),<Out>>$\\
Si $out = write x$ et $x \in dom(Env)$ et $Env(x) \in Ref$ NewOut = $<$"ref:"$+tochar(Env(x)),<Out>>$\\
Si $out = write x$ et $x \in dom(Env)$ et $Env(x) \in {undefined}$ NewOut = $<$"undefined"$,<Out>>$\\

La fonction \emph{tochar(x)} prend un \emph{int} ou une référence en paramètre et la convertis en une séquence de caractère. 
Dans le cas d'une référence, ce qui est imprimé dépend de l'implémentation et n'a pas de garantie d'être
constant entre les plateformes ou exécutions successives (de la même façon qu'en pointeur en C). 

\section{Relations de transition}
\paragraph{Conditions}
$<e,l,s,P,in,out> \rightarrow$ $l$ if cond then $l1$ else $l2$  $\rightarrow <e,l',s,P,in,out>$\\
$l'$ = $l1$ si B[cond] = $true$\\
$l'$ = $l2$ si B[cond] = $false$\\
si B[cond]  = $error$ alors le programme s'arrête en affichant un message d'erreur.

\paragraph{Affectations}
$<e,l,s,P,in,out> \rightarrow$ $l$ ass $l'$ $\rightarrow <e',l',s',P,in,out>$\\
$(e',s')$ = A[[ass]]$ e\: s$\\
si A[[ass]]$ e s$ = $error$ alors le programme s'arrète en affichant un message d'erreur.

\paragraph{Entrée/Sortie} 
$<e,l,s,P,in,out> \rightarrow$ $l$ read x $l'$  $\rightarrow <e',l',s,P,in',out>$\\
$in',e' = $ I[[read x]] $e\: in$\\
si I[[read x]] $e in$ = $error$ alors le programme s'arrète en affichant un message d'erreur.

$<e,l,s,P,in,out> \rightarrow$ $l$ write x $l'$  $\rightarrow <e,l',s,P,in,out'>$\\
$out' = $O[[write x]] $e\: out$
si O[[write x]] = $error$ alors le programme s'arrète en affichant un message d'erreur. 
e' et s' sont les valeurs renvoiée par la fonction d'affectation.

\paragraph{appels de méthodes statiques}
$<e,l,s,P,in,out> \rightarrow x = m(x_1,...,x_n) \rightarrow <e',l',s,P',in,out>$\\
$e'$ = $x_i \rightarrow V[x_i]$ soit l'ensemble des paramètres associés à leur valeur d'appel. \\
$l'$ est le label de la première instruction de la méthode m.\\
$P'$ = $ < l e x <P>>$ Soit la pile $P$ à laquelle on ajoute le label $l$,l'environnement $e$, et la variable de retour $x$\\
La méthode est connue et choisie grâce à la réference m

\paragraph{appels de méthodes dynamique}
$<e,l,s,P,in,out> \rightarrow x = y.m(x_1,...,x_n) \rightarrow <e',l',s,P',in,out>$\\
$e'$ = $x_i \rightarrow V[x_i] \bigoplus this \rightarrow Env(y)$ soit l'ensemble des paramètres associés à leur valeur d'appel auquel
on ajoute la définition du this.\\
$l'$ est le label de la première instruction de la méthode \emph{m/i} ou i est l'entier le plus grand qui soit plus petit ou égal à
$n$ défini par $<n,<...>> = Store(Env(y))$\\
$P'$ = $ < l e x <P>>$ Soit la pile $P$ à laquelle on ajoute le label $l$,l'environnement $e$, et la variable de retour $x$\\
La méthode est connue et choisie grâce à la réference m

\paragraph{appels vers super}
$<e,l,s,P,in,out> \rightarrow x = super.m(x_1,...,x_n) \rightarrow <e',l',s,P',in,out>$\\
$e'$ = $x_i \rightarrow V[x_i] \bigoplus this \rightarrow Env(y)$ soit l'ensemble des paramètres associés à leur valeur d'appel auquel
on ajoute la définition du this.\\
$l'$ est le label de la première instruction de la méthode \emph{m/i} ou i est l'entier le plus grand qui strictement plus petit que 
$n$ défini par $<n,<...>> = Store(Env)$\\
$P'$ = $ < l e x <P>>$ Soit la pile $P$ à laquelle on ajoute le label $l$,l'environnement $e$, et la variable de retour $x$

\paragraph{retour de fonction}
$<e,l,s,P,in,out> \rightarrow x_r \rightarrow <e',l',s,P',in,out>$\\
$P = <l',e'',x,P'>$\\
$e'= e''\bigoplus x \rightarrow e(x_r)$\\
C'est a dire qu'on remplace l'environnement par celui qui était sur la pile, dans lequel
on change la variable de retour par la valeur de retour, et l'on change le label
par le label de retour. 

\paragraph{démarrage du programme}
le programme démarre au début de la fonction 'main' sans paramètres, avec une pile,un environnement et un store vide. C'est à dire : \\
$<e,l,s,P,in,out> \rightarrow main()$ \\
$e = \{\}$\\
$P = \{\}$\\
$s = \{\}$




