\section{Conclusion}
	Ce projet fut pour nous l'occasion de découvrir l'envers
	du décor des compilateurs et interpréteurs, programmes que nous
	utilisons quotidiennement sans jamais savoir comment ils fonctionnent réellement.

	Nous avons aussi vu comment concevoir un langage de programmation, compétence
	indispensable pour tout ingénieur en informatique. 

	Ce fut aussi l'occasion de découvrir le lisp que nous ne connaissions
	que de réputation. Nous poursuirons très certainement l'apprentissage
	de ce language très élégant, et certains membres du groupes se sont
	même mis l'idée en tête d'en écrire un interpréteur plus complet. 

	Nous trouvons cependant dommage qu'il ne nous ai pas été demmandé plus
	tôt de réaliser l'analyseur syntaxique. En effet, réaliser celui ci
	n'est pas très compliqué et permet de trouver des erreurs dans la syntaxe,
	que nos vérificateurs n'avaient pas trouvés. 

	Nous trouvons aussi dommage le manque de ressources écrites sur la syntaxe
	WP. Tout n'était pas clair et parfois l'assistant n'était pas en mesure de
	nous aider. Ce projet était un défi très intéressant il est dommage que 
	les ressources ne soient pas à sa hauteur. 

	Enfin nous ne pouvons ommettre que la charge de travail dépasse largement
	Le cadre d'un cours à 5ects. Ce qui semble d'ailleurs être un problème 
	général des cours d'INFO. Il ne faut pas oublier qu'avec 6 cours et autant
	de groupes différents nous perdons beaucoup de temps à nous organiser et
	qu'il n'est pas facile de trouver des plages horaires qui permettent de
	satisfaire la présence au cours de tout le monde. 

	Nous somme en tout cas satisfait de notre travail et de ce projet et 
	espérons que vous l'êtes aussi. :) 



	


