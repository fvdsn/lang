\chapter{Mode d'emploi du compilateur}
La version exécutable du compilateur et interpréteur Happy se trouve dans le jar exécutable \textit{happy.jar}.
Il prend 2 arguments obligatoire : la grammaire au format BNF et le programme en langage Happy. On peut rajouter \textit{check} à la fin pour vérifier la grammaire en plus.
Pour faire fonctionner les programmes Happy, il faut passer la grammaire \textit{temp.bnf}. Pour vérifier n'importe quelle grammaire WP, il faut indiquer la grammaire
un fichier programme bidon et enfin check.
exemple : \begin{verbatim}
    java -jar happy.jar temp.bnf programme1.happy 
    java -jar happy.jar temp.bnf programme1.happy check
          \end{verbatim}

L'interpréteur sort les informations suivantes dans cet ordre, si le check de la grammaire est demandé, le résultat des tests et la table de précédence.
Ensuite l'arbre sortit par le parser syntaxique après sa restructuration, ensuite la traduction de l'arbre en SLIP et puis (mais il semble que
les retour à la ligne ne soit pas conforme ce qui donne des résultats étranges à la sortie dans le shell) la représentation en code interne SLIP. Et finalement
la sortie du programme interprèté. 


\section{Les erreurs}
Lorsqu'une erreur arrive dans n'importe quelle partie du compilateur l'erreur est affichée (avec plus ou moins de précision) et il s'arrête ensuite.
\section{Les erreurs du vérificateur}
Le vérificateur de grammaire WP évalue les critères les uns après les autres et s'arrète dès qu'un critère n'est pas respecté, en indiquant à la console
l'endroit où cela coince de manière assez explicite.  

\section{Les erreurs du parser lexical}
  Il ne vérifie que deux choses. A la fin de l'analyse si il n'y a pas le même nombre de paranthèse ouvrantes que fermantes, l'erreur \textit{Unexpected end}.
  Il est aussi capable de vérifier si à tout moment il y a trop de paranthèse fermantes, le messages est dès lors très explicite : \textit{Too much ) }.
\section{Les erreurs de l'analyseur syntaxique}
  L'analyseur syntaxique signale trois erreurs. Lorsque deux termes ne peuvent se retrouver côte à côte, lorsqu'il n'arrive pas à trouver une règle pour réduire.


Et lorsqu'il a lu tout les caractères, si il reste plus d'un élément dans la pile une erreur est aussi renvoyé.
\section{Les erreurs du traducteur}
    Les erreurs que renvoient le traducteur sont d'ordre sémantique. Le programme est valide syntaxiquement mais comme notre grammaire permet beaucoup de chose,
Il faut remettre les choses en places avec le traducteur. Les erreurs sont du type : quelquechose est attendu dans ce type d'expr et là c'est pas le cas. 
exemple : textit{Expected id after a set}


\section{Les erreurs de l'interpréteur}
  L'interpréteur reporte les erreurs d'exécutions telles que la division par 0, l'appel à méthode inconnue etc...
  Les erreurs sont clairement identifié dans le code interne structurée et reporté jusqu'au dessus de la pile d'exécution. Cela permet de retracer l'erreur
sans trop de difficulté. 
Voici un exemple : 
\begin{verbatim}
Error divide by 0
	 in Cexpr a#1 / b#2
	 at Ass A_0#3 := a#1 / b#2
	 at CmdStmt [ lab9 : A_0#3 := a#1 / b#2 ; go to lab8]
	 at divide/-1
	 at Call divide
	 at CmdStmt [ lab2 : A_0#3 := divide(#4, #5) ; go to lab1]
	 at main/-1
\end{verbatim}
