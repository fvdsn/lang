\section{Vérificateur de grammaire WP}
	\subsection{BNF Parser}
	Le point de départ du vérificateur de grammaire est le parser BNF. Ce parser permet d'avoir sous une forme utilisable que nous détaillerons 
plus loin, les règles contenue dans le fichier. Le loader charge le fichier et le lit ligne par ligne. Chaque ligne contient l'ensemble des règles
de production d'un non terminal, séparées par des barres. Les non terminaux sont entre '<','>' et les terminaux sont entre guillements. On peut
échapper les guillements ou les comparateurs par un backslash. 
Voici un exemple de fichier BNF accepté par le parseur : 

\begin{verbatim*}
	  <E> ::= <T> | '\\' <T> | <T> '+' <E>
	  <T> ::= <F> | <F> '*'<T>
	  <F> ::= '\''
\end{verbatim*}

	\subsection{Représentation Java de la grammaire et de ses éléments}
		Toutes les classes représentant la grammaire et le parsing du bnf se trouvent dans le package \emph{happy.parser.bnf}
		\subsubsection{Term} Le \emph{Term} est la classe représentant le symboles terminaux et non terminaux de la syntaxe. \emph{Term}
		représente aussi les terminaux renvoyés par l'analyseur lexical, et les noeuds des différents arbres syntaxiques. Cela permet d'éviter
		les conversions inutiles dans nos différents algorithmes. Sa méthode qui nous concerne ici est getType() qui renvoit le nom du symbole
		(Number , Identifier , ...)
		\subsubsection{CatList} La \emph{Catlist} est une liste de \emph{Term} représentant une partie droite de production. 
		\subsubsection{Rule}  La \emph{Rule} représente une règle de production. getName() renvoie le \emph{Term} de la partie gauche, 
		et getOrList() renvoie une liste de \emph{Catlist}

	\subsection{Architecture}
	Le vérificateur de grammaire est implémenté comme une série de test
	statique correspondant aux conditions de Weak Precedence.
	
	Tous ces tests prennent en paramètre la grammaire (une liste de "Rules")

	Un test global - \emph{CheckAll} - se charge de tester l'ensemble de ces tests.
	Toutes les classes et méthodes correspondantes se trouvent dans le package \emph{happy.checker}

	\subsection{Test de symbole vide (\emph{CheckLambda})}
		Ce test s'assure qu'aucune règle n'a de production vide (lambda).
		Ce test parcourt simplement l'arbre de la définition grammaire et
		vérifie l'absence de terminaux lambda. Ce test vérifie aussi l'absence
		de Non terminaux n'ayant pas de règles de productions. 

	\subsection{Test de conflit de préférence (\emph{CheckPrecedence})}
		Ce test vérifie qu'il n'y ait pas de conflit entre les précédence
		de symbole. Le seul conflit autorisé est entre $\lessdot$ et $\doteq$ qui donne
		la relation de précédence $\dot{\leq}$

		Pour faire cela, le test construit deux ensemble pour chaque non-terminal :
		First et Last. Ceux-ci permettront de déterminer très facilement les précédences
		par la suite. Soit un non terminal $A$ : $First(A)$ contient l'ensemble des terminaux et non terminaux 
		pouvant se trouver sur l'extrème gauche d'une production de $A$. $Last(A)$ contient l'ensemble
		des terminaux et non terminaux pouvant se trouver sur l'extrème droite d'une production de $A$. 

		Pour calculer Tous les First on utilise l'algorithme suivant :
		\begin{enumerate}
			\item Pour tout non terminal $A$ de la grammaire : On identifie les règles correspondantes.
			Pour chacune d'entre elles on identifie si elles commencent par un terminal, et si
			c'est le cas on les ajoute à $First(A)$. 
			\item Pour tout non terminal $A$ de la grammaire : On identifie les règles correspondantes,
			Pour chacune d'entre elles, on identifie si elles commencent par un non terminal $B$.
			Si c'est le cas on ajoute $B$ et $First(B)$ à $First(A)$. 
			\item on répète l'étape précédente tant que les $First$ changent.
		\end{enumerate}
		L'algorithme pour trouver les Last est identique à ceci près qu'on examine les cotés droits des règles.
		
		Une fois que l'on a ces deux ensembles on peut calculer la table de précedence, grace à l'algorithme suivant :
		\begin{enumerate}
			\item On initialise tous les éléments de la table à '$Nothing$'
			\item On identifie toutes les parties droites, et pour chacune d'entre elles on identifie tous les
			symboles côte à côte $..XY..$
			\item On ajoute $X\doteq Y$ dans la table.
			\item Si $X$ non terminal : Pour tout symbole $s$ dans $Last(X)$, Pour tout terminal $t$ dans $First(Y)$
				on met $s\gtrdot t$ dans la table. Si $Y$ est terminal on met $s\gtrdot Y$ dans la table.
			\item Si $Y$ non terminal : Pour tout symbole $s$ dans $First(Y)$, on met $X \lessdot s$ dans la table.
			\item Si à un moment on met une relation dans la table là ou précédemment il n'y avait pas '$Nothing$',
				on a un conflit. Les conflits impliquant $\lessdot$ , $\doteq$ et $\dot{\leq}$ sont résolus en insérant $\dot{\leq}$ dans
				la table. Les autres conflits sont marqués en tant que conflits et reportés. 
			\item Si il y a des conflits non résolus dans la table alors le Test échoue, la grammaire n'est pas valide
			Weak Precedence.
		\end{enumerate}
	\subsection{Preuve de l'algorithme}
		Cet algorithme nous vient de l'ouvrage \emph{Parsing techniques - A practical guide de Dick Grune et Ceriel J.H Jacobs}
		\subsubsection{$\doteq$} Le critère utilisé par notre algorithme est identique à celui donné dans le cours.
		\subsubsection{$\lessdot$} Le cours donne le critère suivant : On a $X\lessdot Y$ si pour $A \Longrightarrow..XC..$ on a $C \overset{+}{\Longrightarrow} Y..$ On constate
		que $First(C)$ donne tous les symboles terminaux et non terminaux tels que $C \overset{+}{\Longrightarrow} Y..$ le critère utilisé est donc
		identique.
		\subsubsection{$\gtrdot$} Le cours donne le critère suivant : On a $X\gtrdot y$ si pour $A\Longrightarrow..BC..$ on 
		a $B \overset{+}{\Longrightarrow} ..X$ et $C \overset{*}{\Longrightarrow} y..$
		$Last(B)$ donne tous les symboles tels que $B \overset{+}{\Longrightarrow}..X$ et $First(C)$ donne tous les symboles tels que 
		$C \overset{+}{\Longrightarrow} Y..$ 
		on en prend tous les terminaux et C lui même si terminal pour avoir l'ensemble tel que $C \overset{*}{\Longrightarrow} y..$ Le critère utilisé
		est donc identique. 

	\subsection{Test de Suffixe}
		Ce test vérifie que les relations de suffixe sont respectées. C'est à dire :
		Pour tout couple de règles $Z \Longrightarrow \beta$, $X \Longrightarrow \alpha Y \beta$,
		on vérifie que $\neg(Y \doteq X)$ , $\neg(Y \lessdot X)$ et $\neg(Y \dot{\leq} X)$.
		
		Pour ce faire nous utilisons la méthode suivante : 
		Pour chaque règle $Z \Longrightarrow \beta$, pour toute autre règle $X \Longrightarrow \alpha$ on regarde si on peut matcher $\beta$ 
		comme suffixe de $\alpha$.
		Dans l'affirmative, on identifie le symbole juste avant le suffixe. Nous pouvons a ce moment vérifier la
		condition décrite expliquée précédemment. Si elle n'est pas vérifiée. le Test échoue, la grammaire n'est pas valide.

	\subsection{Test de Cycles}
		Un cycle est definit formellement par tout ensemble de règles de productions donnant $E \overset{+}{\Longrightarrow} \alpha E\beta$.
		ou $\alpha$ et $\beta$ sont 'nullables'. 
		Comme en WP il ne peut y avoir de productions vides, on vérifie $\neg(E \overset{+}{\Longrightarrow} E)$. Pour ce faire on 
		crée un graphe directionnel ayant pour noeuds les symboles non terminaux et pour arcs les règles de productions de type
		$A\Longrightarrow B$. On vérifie ensuite l'absence de cycles dans ce graphe via un algorithme adéquat.

	\subsection{Spécifications des méthodes importantes}
{\small
\begin{verbatim}


 	/**
	 * 
	 * @param rules The rules list
	 * @param table The precedence table
	 * @return a list of tuple (triple) 
	 * that contains the two conflicting rules and 
	 * the suffix the cause the conflict
	 */
	public static List<RulesTuple> checkSuffix(List<Rule> rules, 
	Hashtable<Term,Hashtable<Term,String>> table)
	
	/**
	 * Regarde si la grammaire n'a pas de conflits de précédence.
	 * Affiche la table des précédence à la sortie standard.
	 * @param grammar Une grammaire cohérente : 
	 * tous les non terminaux ont au moins une règle. 
	 * @return true si la grammaire est valide, false sinon.
	 */
	public static boolean checkPrecedence(List<Rule> grammar)
	
	/**
	 * Met une relation de précédence dans la table de précédence et
	 * résout les conflits. Affiche les détails de l'erreur à la sortie
	 * standard en cas de conflits. Si le conflit est entre <. ou <.= ou =. 
	 * celui ci est résolu en insérant un <.= 
	 * 
	 * @param table la table de précédence
	 * @param X	le premier Terme
	 * @param Y le second Terme
	 * @param rel la relation de précédence ( EQ,LE,GE,LEQ,ERROR,NOTHING)
	 * @param rule la règle d'ou provient X et Y.
	 * @return true si il n'y a pas de conflits, false sinon. 
	 */
	public static boolean tableSet(Hashtable<Term,Hashtable<Term,String>> 
	table, Term X, Term Y, String rel, Rule rule)
	
	/**
	 * Calcule l'ensemble First pour chaque Terme correspondant aux Termes
	 * pouvant se trouver à l'extrème gauche de celui ci. 
	 * @param grammar la grammaire
	 * @return une Hashtable qui fait correspondre à chaque Terme son set First.
	 */
	public static Hashtable<Term,Set<Term>> FirstSet(List<Rule> grammar)
	
	/**
	 * Calcule la table de précédence WP et si celle ci contient des conflits.
	 * En cas de conflits, ceux ci sont affichés à la sortie standard.
	 * @param grammar la grammaire
	 * @param table une Hashtable à deux dimensions dont l
	 * es clefs sont tous les couples
	 * 			de termes existant dans grammar. 
	 *  Tous les éléments de table sont initialisés
	 * 			à NOTHING.
	 * @return true si il n'y a pas de conflits dans la table, false sinon. 
	 */
	public static  boolean precTable(List<Rule> grammar, 
	Hashtable<Term,Hashtable<Term,String>> table)
	
	/**
	 * Vérifie que la liste ne contient pas d'expensions vides,
	 * en cherchant les termes 'lambda' dans tous les termes de toutes
	 * les règles. 
	 * @param grammar la grammaire
	 * @return  true si la grammaire ne contient pas de terme 'lambda',
	 * 			false sinon.
	 */
	public static boolean checkLambda(List<Rule> grammar)


\end{verbatim}
}
\section{Test du vérificateur}
Nous avons testé notre vérificateur de grammaire avec les grammaires fournie sur icampus 
pour finalement tester avec notre grammaire. Ici je ne montrerai que le test final pour notre grammaire. 

\subsection{Premier Test}
{\tiny
\begin{verbatim}
<A> ::= <A> 'a'
<A> ::= <B>
<B> ::= 'a'
<B> ::= <A> <B>

Check conflit suffix : 
----------------------------
A   A
<.
----------------------------
----------------------------
B   A
<=
----------------------------
conflit avec a:true
A ::= A:false a:true
B ::= a:true 

conflit avec B:false
B ::= A:false B:false
A ::= B:false 
\end{verbatim}
}

Validity grammar : false


Ce test concorde bien avec l'exmple donné
{\tiny
\begin{verbatim}
Grammaire non WP :
règle 1 : B --> a 
règle 2 : A --> A a
A <. B
Grammaire non WP :
règle 1 : A --> B 
règle 2 : B --> A B
A <. A

\end{verbatim}
}


\subsection{Second Test}:
{\tiny
\begin{verbatim}
<E> ::= <T> 
<E> ::= '+' <T> 
<E> ::= <E> '+' <T> 
<T> ::= <F> 
<T> ::= <T> '*' <F> 
<F> ::= 'x' 
<F> ::= '(' <E> ')' 
\end{verbatim}
}
C'est bien une grammaire WP :
{\tiny
\begin{verbatim}
Precedence Table
    |(   |)   |*   |+   |T   |E   |F   |x   |
     ========================================
(   | <. |    |    | <. | <. | <= | <. | <. |
     ----------------------------------------
)   |    | .> | .> | .> |    |    |    |    |
     ----------------------------------------
*   | <. |    |    |    |    |    | .= | <. |
     ----------------------------------------
+   | <. |    |    |    | <= |    | <. | <. |
     ----------------------------------------
T   |    | .> | .= | .> |    |    |    |    |
     ----------------------------------------
E   |    | .= |    | .= |    |    |    |    |
     ----------------------------------------
F   |    | .> | .> | .> |    |    |    |    |
     ----------------------------------------
x   |    | .> | .> | .> |    |    |    |    |
     ----------------------------------------

No conflicts in table
Validity precedence : true
Check conflit suffix : 
Validity grammar : true
\end{verbatim}
}

\subsection{Troisième Test}
Dernière grammaire qui est normalement fausse :
{\tiny
\begin{verbatim}
E --> T 
E --> + T 
E --> T + E 
T --> F 
T --> F * T 
F --> x 
F --> ( E ) 

Precedence Table
    |(   |)   |*   |+   |T   |E   |F   |x   |
     ========================================
(   | <. |    |    | <. | <. | .= | <. | <. |
     ----------------------------------------
)   |    | .> | .> | .> |    |    |    |    |
     ----------------------------------------
*   | <. |    |    |    | .= |    | <. | <. |
     ----------------------------------------
+   | <. |    |    | <. | <= | .= | <. | <. |
     ----------------------------------------
T   |    | .> |    | .= |    |    |    |    |
     ----------------------------------------
E   |    | .= |    |    |    |    |    |    |
     ----------------------------------------
F   |    | .> | .= | .> |    |    |    |    |
     ----------------------------------------
x   |    | .> | .> | .> |    |    |    |    |
     ----------------------------------------

No conflicts in table
Validity precedence : true
Check conflit suffix : 
----------------------------
E   +
.=
----------------------------
conflit avec T:false
E ::= +:true T:false
E ::= T:false 

Validity grammar : false
\end{verbatim}
}
