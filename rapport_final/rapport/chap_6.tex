\section{Analyse syntaxique}
	\subsection{Grammaire WP du langage} 
		\begin{verbatim}
			<l0> ::= <l_block>  ')' | 'l_id'
			<l_block> ::= '(' <l_list>
			<l_list> ::= <l0> | <l0> <l_list>
		\end{verbatim}
		Étant donné les nombreux déboires et retournement de situations
		pour rendre notre syntaxe WP nous avons décidé d'utiliser une
		autre approche qui consiste à seulement différencier les différents
		niveaux d'imbrication de parenthèses et les 'identifiants' qui 
		comprennent les identifiants, mais aussi les mots réservés et les 
		nombres. Ceux ci sont désignés par 'l_id' dans la grammaire, mais
		en pratique, les symboles gardent leur type donné par l'analyseur
		lexical. Le gros du travail est donc délégué aux étapes ultérieures.
	\subsection{Problèmes rencontrés}
		Le premier problème venait du flou entourant la definition de cycle.
		Nous avions travaillé sur une mauvaise définition beaucoup trop 
		restrictive qui nous a amené a différencier chaque niveau d'imbrication.
		Comme cela n'était pas possible avec la syntaxe de base nous avons utilisé
		la syntaxe minimaliste présentée ci-dessus. Lorsque nous nous sommes rendu
		compte de notre erreur, nous avons continué dans notre approche. En effet
		la validation de la syntaxe n'est pas spécialement plus compliquée au niveau
		de l'arbre syntaxique. 
		
		Le second problème est un conflit de précédence au niveau des parenthèses
		qui a été résolu en séparant les paires de parenthèses en deux rêgles
		de production séparées. 
	\subsection{Code de l'analyseur syntaxique}
