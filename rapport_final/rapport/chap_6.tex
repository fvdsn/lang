
\chapter{Analyse syntaxique}
  \section{Grammaire WP du langage} 
		\begin{verbatim}
			<l0> ::= <l_block>  ')' | 'l_id'
			<l_block> ::= '(' <l_list>
			<l_list> ::= <l0> | <l0> <l_list>
		\end{verbatim}
		Étant donné les nombreux déboires et retournement de situations
		pour rendre notre syntaxe WP nous avons décidé d'utiliser une
		autre approche qui consiste à seulement différencier les niveaux d'imbrication de parenthèses 
		et les 'identifiants' qui 
		comprennent les identifiants, mais aussi les mots réservés et les 
		nombres. Ceux ci sont désignés par 'l\_id' dans la grammaire, mais
		en pratique ces symboles gardent leur type donné par l'analyseur
		lexical. Le gros du travail est donc délégué aux étapes ultérieures.
	\section{Problèmes rencontrés}
		Le premier problème venait du flou entourant la definition de cycle.
		Nous avions travaillé sur une mauvaise définition beaucoup trop 
		restrictive qui nous a amené a différencier chaque niveau d'imbrication.
		Comme cela n'était pas possible avec la syntaxe de base nous avons utilisé
		la syntaxe minimaliste présentée ci-dessus. Lorsque nous nous sommes rendu
		compte de notre erreur, nous avons continué dans notre approche. En effet
		la validation de la syntaxe n'est pas spécialement plus compliquée au niveau
		de l'arbre syntaxique. 
		
		Le second problème est un conflit de précédence au niveau des parenthèses
		qui a été résolu en séparant les paires de parenthèses en deux rêgles
		de production séparées.
	\section{Code de l'analyseur syntaxique}
	Voici le code de l'analyseur syntaxique, la méthode principale est parse, qui fait
	appel aux sous méthodes \emph{shift} et \emph{reduce} L'invariant de parse est le
	suivant : Soit S la stack : et L la liste de symboles non encore lus :
	\begin{itemize}
		\item soit S contient une poignée à son sommet.
		\item soit S ne contient pas de poignée et L est non vide. 
		\item soit S contient l'unique symbole start et L est vide et la syntaxe est parsée.
		\item soit le programme n'est pas valide. 
	\end{itemize}
	parse() passe de l'un de ces cas à l'autre avec shift et reduce pour consommer L et arriver au
	symbole de départ, ou à une erreur auquel cas le programme s'arrète. 

\section{Analyse syntaxique}
	


  {\tiny  \begin{verbatim}
/*
 * @pre :~
 * @post: A terminal is read from the lexical parser and put ont
 * the top of the stack, and returns true.
 * If there is no terminal left to read, it will do nothing and return 
 * false
 */
public boolean shift(){
  if(next == null && lexParser.hasNext()){
    System.out.println("coucou");
    stack.push(lexParser.next());
    if(lexParser.hasNext()){
      next = lexParser.next();
    }
    return true;
  }else if(next != null){
    stack.push(next);
    next = lexParser.next();
    return true;
  }
  return false;
}
/*
 * @pre : There is a symbol on the stack.
 * @post : ~
 * @returns : 
 * Returns the precedence relation between the symbol with index
 * i and i+1 on the stack. If i is equal to the stack size -1 it
 * returns the precedence relation with the symbol that will be
 * placed on the stack on the next shift.
 */
public String prec(int i){ ... }
/*
 * @pre : ~
 * @post : Prints the stack and the next element. 
 */
public void printStack(){ ... }
/* 
 * @pre: ~
 * @post: ~
 * @return: true if there is only the start symbol left on the stack, and 
 * nothing left to read on the lexical parser.
 */
public boolean done(){ ... }

/*
 * @pre : There is a symbol on the stack.
 * @post :
 * Tries to match the stop of the stack with the right side of a rule.
 * It will try to match <.= ... > handles first. If it cannot match
 * the first <. ... .> handle found, it will stop and return false.
 * If it finds a match the handle is removed from the stack, put as
 * the child of a new Term corresponding to the left hand of the matching
 * rule, and that Term is put on the top of the stack.
 * @return : returns true if could reduce, false if it couldn't. 
 */
public boolean reduce(){
  int i = stack.size() -1; /* beginning of the handle */
  boolean hard = false;  /*it has tried to match a < ... > handle */
  boolean match = true;  
  while(i >= 0){
    int matchsize = stack.size()-i;
    if(   prec(i-1).equals(CheckPrecedence.LE)
      ||  prec(i-1).equals(CheckPrecedence.LEQ)){
      if(prec(i-1).equals(CheckPrecedence.LE)){
        hard = true;
      }
      for(Rule r:grammar){
        for(CatList c:r.getOrList()){  /*we iterate over rules */
          List<Term> L = c.getTermList();
          if(L.size() != matchsize){
            continue;
          }
          int j = 0;
          match = true;
          while(j < matchsize){  /*matching the handle with the rule*/
            if(!L.get(j).equals(stack.get(i+j))){
              match = false;
              break;
            }
            j++;
          }
          if(!match){
            continue;
          }
          Term R = new TermImpl(r.getLeftSide().getType(),false);
          j = 0;
          while(j < matchsize){ /*removing from the stack and adding to the child list */
            R.getChildList().add(stack.remove(i));
            j++;
          }
          stack.push(R);
          return true;
        }
      }
    }
    if(hard){
      return false;
    }
    i--;
  }
  return false;
}
/*
 * @pre : the object has been correctly initialized with a lexical
 * parser and a precedence table.
 * @post : Tries to parse the input, if successful puts the result in tree and
 * returns true. returns false and prints an error message otherwise. 
 */
public boolean parse(){
  while(shift()){
    while(prec(stack.size()).equals(CheckPrecedence.GE)){
      if(!reduce()){
        if (done()){
          tree = stack.get(0);
          return true;
        }
        System.out.println("Syntax Error : Could not reduce");
        printStack();
        return false;
      }
      if(done()){
        tree = stack.get(0);
        return true;
      }
    }
  }
  if (done()){
    tree = stack.get(0);
    return true;
  }else{
    System.out.println("Syntax Error : Program too long");
    return false;
  }
}
  \end{verbatim}}

