\documentclass[11pt,a4paper]{report}
\usepackage{ifpdf}
\usepackage[utf8]{inputenc}
\usepackage[francais]{babel}
\usepackage[french]{varioref} 
\usepackage[pdftex]{graphicx}
\usepackage{listings}
\usepackage{color}
\usepackage{amssymb}
\usepackage{amsmath}
\usepackage{moreverb}

%\usepackage{floatflt}
\usepackage{lscape} %afficher une partie en paysage


%%% Pour du code source %%%%
 \definecolor{colKeys}{rgb}{0.5,0,0.33} 
 \definecolor{colIdentifier}{rgb}{0.16,0,1} 
 \definecolor{colComments}{rgb}{0.25,0.5,0.37} 
 \definecolor{colString}{rgb}{0.6,0.1,0.1} 
 \definecolor{shadow}{rgb}{0.5,0.5,0.5} 
 
 \lstset{ 
 basicstyle=\ttfamily\small,
 identifierstyle=\color{colIdentifier},
 keywordstyle=\color{colKeys},
 stringstyle=\color{colString},
 commentstyle=\color{colComments}
 }
 \lstset{language=python}



\begin{document}

%\tableofcontents
\thispagestyle{empty}
\begin{center}
\begin{tabular}{lr}
\begin{minipage}[l]{0.4\textwidth}
\begin{flushleft}
\includegraphics[scale=0.2]{logo.png}
\end{flushleft}
\end{minipage}
&
\begin{minipage}[r]{0.4\textwidth}
\begin{flushright}
\begin{tabular}{l}
Van De Walle Bernard  (A)\\
Francois Thibault (B) \\
Van Der Essen Frédéric (C)
\end{tabular}
\end{flushright}
\end{minipage}
\end{tabular}
\end{center}

\vspace{6.5cm}

\begin{center}
\textsc{INGI2132: Langages et traducteurs}
\end{center}

\bigskip

\begin{center}
{\Huge Rapport Final}
\end{center}

\vspace{7.5cm}
\begin{center}
		\textbf{Prof.} B. Le Charlier\\
\end{center}

\vspace{1.5cm}

\newpage

\begin{figure}[h!]
\begin{center}
\includegraphics[scale=0.65]{xkcd.png}
\end{center}
\end{figure}
It is now possible with the HAPPY-) programming language !
\newpage

\section{Introduction}
	Voici le rapport final de notre projet où il nous était demandé de
	réaliser un interpréteur avec comme contrainte l'utilisation d'une
	syntaxe WP. 

	Nous pensons avoir complété l'entièreté du projet. Il reste cependant
	un oubli au niveau du vérificateur de grammaire : Celui ci ne vérifie
	pas la présence de Symboles non terminaux intermédiaires inutiles. 

	Le rapport suit à la lettre le plan donné dans les consignes. 


\chapter{Présentation de Happy:-)}
\section{Introduction}
La langage happy, appelé ainsi parce qu'il est très permissif au niveau des caractères permit dans les identifiants comporte de nombreuses autres particularités. Tout d'abord, le langage ressemble très fort au LISP où tout est atom ou liste.
Mais contrairement au LISP, notre langage est impératif et orienté objet. Un programme est une liste de méthode, une méthode est une liste dont le premier élément est le mot réservé fun le second une list d'argument et le dernier une liste de commande. Même principe pour le while et le if. Dans ce langage tout est fonction, dans le sens que toute instruction renvoie une valeur, y compris les if et le while qui renvoient 0, une fonction qui n'a pas d'instruction return renvoie null, le write renvoie la valeur qu'elle vient d'imprimer etc...



Les conditions aurait aussi renvoyé une valeur si le langage SLIP dans lequel notre langage est traduit le permettait. 

\section{Grammaire de départ}
Cette grammaire est la grammaire exhaustive du langage si on rajoute que les id peuvent être formé de tout les caractères UTF-16 sauf des caractères réservés et qu'ils ne doivent pas être égale à un mot réservé.

\begin{verbatim}
<Program> ::= ( <Prog_list> )
<Prog_list> ::=  <Meth_or_fun> | <Prog_list> <Meth_or_fun>	
<Meth_or_fun> ::=  <Method> | <Function>

<Function> := ( fun ( <Arglist> ) <Instr_list> ) 
<Arglist> ::=  id | id <Arglist>
<Method> ::= ( <Method_int> ( <Arglist> ) <Instr_list> )
 
<Instr_list> ::=  <Instr> | ( <Instr_list_np> ) 
<Instr_list_np> ::=  <Instr> | <Instr> <Instr_list_np>
<Instr> ::=  <Conditional> | <While_block> | <Call> | ( return <Expr> ) 

<Conditional> ::= ( if <Cond> <instr_list> <instr_list>  )  
<Conditional> ::= ( if <Cond> <instr_list> )
<While_block> ::= ( while  <Cond> <Instr_list>  ) 
<call> ::=  <User_call> | <Method_call> | <Builtin_call> 
<Builtin_call> ::=  <Assignment> | <Read_call> | <Write_call> | <Arithmetic_call>
<Assignment> ::= ( set id <Expr> ) 
<Assignment> ::= ( set <Id_int> <Expr> ) | ( set <This_int> <Expr> )
<Read_call>  ::= ( read )
<Write_call> ::= ( write <Expr> )
<Arithmetic_call> ::=  <Binary> | <Unary>
<Binary> ::= ( <Bin_id> <Expr> <Expr>  )
<Unary>  ::= ( <Un_id> <Expr>  )
<User_call>  ::= ( id  <Expr_list_np> ) | ( id ) 
<Expr_list_np> ::=  <Expr> | <Expr> <Expr_list_np>
<Method_call>  ::= ( <Id_id> <Expr_list_np>  ) |  ( <Id_id> )   
<Method_call>  ::= ( <Super_id>  <Expr_list_np>  ) | ( <Super_id>  )
<Method_call>  ::= ( <This_id>  <Expr_list_np>  ) | ( <This_id> )
<Expr> ::= number | null | true | false | this | id 
<Expr> ::= <Id_int> | <This_int> | <Instr>
<Cond> ::= <Rel> | ( <Log_bin_op> <Cond> <Cond> )
<Cond> ::= ( <Log_un_op> <Cond> <Cond> )
<Rel>  ::= ( <Rel_op> <Expr> <Expr> ) 
<Rel_op> ::= <= | >= | > |  < | =
<Log_bin_op> ::= and | or
<Log_un_op> ::= !
<Bin_id> ::= + | - | * | / | %
<Un_id> ::= neg
<Id_int> ::= id . number
<This_int> ::= this . number
<Super_id> ::= super . id
<This_id> := this . id 
<Id_id> ::= id . id
<Method_int> ::= method . number
\end{verbatim}
Cette grammaire n'est pas wp, mais elle exprime très bien ce qui est syntaxiquement correcte dans ce que nous avons réellement implémenté. 

Voici quelque bout de code permis par cette grammaire :

\begin{verbatim}
 (while (<= (3 * i) (9)) (write (set i (+ i 1))))
  
  (if (! (= i 9)) (return true) (return false))

 (fun (++ i) (return (+ i 1)))

  (set i (++ i))
\end{verbatim}





 
\section{L'analyseur Lexical}

L'analyseur Lexical a pour fonction de sortir une suite de jetons concrets et de assez haut niveau, permettant de simplifier la tache du vérificateur de grammaire. Plus concrètement, il s'agit dans un premier temps de transformer un fichier texte en une suite de caractère, et de les parser les un après les autres. Chaque caractère donnera lieu à une action précise, par exemple les caractères spéciaux tel que les parenthèses et les points, etc... donnent lieux à un jeton entier.
Les caractères qui ne sont pas des caractères spéciaux vont quand à eux remplir un buffer , qui sera rempli des qu'un caractère spécial ou délimiteur sera rencontré. Une fois qu'un caractère de ce type est rencontré, on procédera à l'analyse de ce que contient le buffer, et selon le cas on en déterminera le type du jeton à donner.

\subsection{Entrée de l'analyseur lexical}

Il est important de préciser correctement ce que l'analyseur lexical doit prendre en entrée. Par soucis de facilité, nous avons décidé de nous restreindre aux caractères ascii 8 bits, et plus particulièrement, toutes les lettres ( miniscule et majuscule ), les chiffres, les opérateurs *,+,/-, le symbole d'égalité, les parenthèses, les crochets ( [] ) ,  et le point.

Ensuite il est nécessaire de définir les différents symboles de base que l'analyseur lexical va passer à l'analyseur syntaxique.

Voici donc les différentes catégories que l'analyseur lexical va sortir :

\begin{itemize}
	\item NUMBER  : Représente un ou plusieurs chiffres à la suite.
	\item ID : Représente une suite de caractères qui n'est pas un mot réservé.
	\item UNARY_OP : Représente un opérateur unaire, tel que l'opérateur NOT : !
	\item BINARY_OP : représente un opérateur binaire, tel l'opérateur +
	\item *RESERVED_WORD* : représente un mot réservé. En fait la catégorie sera le nom du mot réservé (d'où les étoiles ), par exemple FUN
	\item *BEFORE_AFTER* : Représente ce qu'il y a avant et après un point. BEFORE et AFTER peuvent valoir : THIS, ID, INT
\end{itemize}

Finalement, il est de bon aloi de préciser que les espaces, les caractères de tabulations et de nouvelle ligne servent évidemment de caractère de séparation. Il n'est par contre pas nécessaire de placer un espace après une parenthèse par exemple, vu qu'il est assez naturel que ce qui suit sera le début d'un nouveau symbole.

Pour ce qui est des commentaires, nous avons introduit les symboles de crochet ( [] ). Tout ce qui se trouvera entre ces deux crochets ( et les crochets eux-même compris ), sera purement et simplement ignoré. Les commentaires peuvent donc s'étendre sur plusieurs lignes sans aucun problème.

Notre analyseur lexical ne renvoie plus de jetons à partir du moment où on atteind la fin du fichier. Il n'est pas nécessaire de terminer le fichier d'une manière particulière.

\subsection{Symboles terminaux de bases}

\subsection{Grammaire formelle}

\subsection{Représentation des symboles lexicaux}

Les Symboles lexicaux sont représenté à l'aide d'un objet de type LexicalTerm. Cet objet contient deux informations nécessaires. La première est la catégorie du symbole lexical, comme défini un peu plus haut. La seconde information est ce que contient concrètement le symbole lexical.
Afin de clarifier cela, un petit exemple s'impose : 
Imaginons que l'analyseur lexical découvre un identifiant pour le mot  "valeura" dans le fichier source. Dans ce cas, il créera un objet LexicalTerm contenant l'information suivante :
\begin{itemize}
	\item TYPE : ID
	\item CONTENT : valeura
\end{itemize}

Dans le cas où maintenant le mot réservé fun est rencontré, l'objet LexicalTerm alors créé contiendra : 
\begin{itemize}
	\item TYPE : fun
	\item CONTENT : fun
\end{itemize}

\subsection{Spécification rigoureuse des méthodes publiques de l'analyseur lexical}

De manière concrète, l'analyseur lexical a été réalisé en implémentant l'interface Iterator et Iterable. Les méthodes publiques seront dons les méthodes propres à ces interfaces . Les méthodes utilisées par l'analyseur syntaxique sont donc :

La méthode hasNext()
\begin{verbatim}
    /**
     * Returns true if the iteration has more elements.
     * (In other words, returns true if next() would
     * return an element rather than throwing an exception.)
     *
     * @return true if the iteration has more Term
     */
		
		public boolean hasNext()
\end{verbatim}


La méthode next()
\begin{verbatim}
    /**
     * @return the next Term in the iteration
     * @throws NoSuchElementException if the iteration has no more Term ( the end of the file is reached ).
     */

		 public Term next()
\end{verbatim}


\subsection{principe d'implémentation}

Le principe d'implémentation est assez simple. Lorsque l'analyseur syntaxique appelle la méthode next() , l'analyseur lexical va lire les caractères suivants dans le fichier, jusqu'a pouvoir déterminer le prochain token.

Tant qu'un caractère de séparation n'est pas reconnu ( un espace, une parenthèse,...), le caractère suivant est lu et placé dans un buffer. Lorsqu'un caractère de séparation est finalement rencontré, ce qui a été mis précedemment dans le buffer devient le prochain jeton, et le caractère de séparation deviendra celui d'après ( sauf si il s'agit d'un espace , d'une nouvelle ligne, ou d'un caractère de tabulation ).

\subsection{Tests de l'analyseur lexical}
Le but de ces tests est de voir les erreurs que renvoie l'analyseur lexical, et surtout de les comprendre.

\subsubsection{Test avec entrée valide}
Voici un petit programme qui est accepté par l'analyseur lexical, il n'affiche donc pas d'erreur, et le programme se déroule bien.

\begin{verbatim}
(
	(fun (main) (
		(set i 0)
		(while (<= i 10)
			(
				(write i)
				(set i (+ i 1))
			)
		)
	))
)
\end{verbatim}

\subsubsection{Tests avec entrée invalide}


\paragraph Test d'un programme ayant une parenthèse fermante de trop ( à la fin ) :
\begin{verbatim}
(
	(fun (main) (
		(write 42)
	))
)
)
\end{verbatim}

Sortie de l'analyseur lexical :

\begin{verbatim}
	Too much ) 
\end{verbatim}

L'analyseur lexical a donc bien détecté qu'il y a trop de parenthèses fermantes.

\paragraph De même, si trop peu de parenthèses fermantes sont présente à la fin du fichier, l'analyseur lexical le détecte :

\begin{verbatim}
(
	(fun (main) (
		(write 42)
	))
\end{verbatim}

En sortie :

\begin{verbatim}
	Unexpected end
\end{verbatim}


\include{chap_5}
\section{Analyse syntaxique}
	\subsection{Grammaire WP du langage} 
		\begin{verbatim}
			<l0> ::= <l_block>  ')' | 'l_id'
			<l_block> ::= '(' <l_list>
			<l_list> ::= <l0> | <l0> <l_list>
		\end{verbatim}
		Étant donné les nombreux déboires et retournement de situations
		pour rendre notre syntaxe WP nous avons décidé d'utiliser une
		autre approche qui consiste à seulement différencier les différents
		niveaux d'imbrication de parenthèses et les 'identifiants' qui 
		comprennent les identifiants, mais aussi les mots réservés et les 
		nombres. Ceux ci sont désignés par 'l_id' dans la grammaire, mais
		en pratique, les symboles gardent leur type donné par l'analyseur
		lexical. Le gros du travail est donc délégué aux étapes ultérieures.
	\subsection{Problèmes rencontrés}
		Le premier problème venait du flou entourant la definition de cycle.
		Nous avions travaillé sur une mauvaise définition beaucoup trop 
		restrictive qui nous a amené a différencier chaque niveau d'imbrication.
		Comme cela n'était pas possible avec la syntaxe de base nous avons utilisé
		la syntaxe minimaliste présentée ci-dessus. Lorsque nous nous sommes rendu
		compte de notre erreur, nous avons continué dans notre approche. En effet
		la validation de la syntaxe n'est pas spécialement plus compliquée au niveau
		de l'arbre syntaxique. 
		
		Le second problème est un conflit de précédence au niveau des parenthèses
		qui a été résolu en séparant les paires de parenthèses en deux rêgles
		de production séparées. 
	\subsection{Code de l'analyseur syntaxique}

%7
\section{Traduction du programme en code interne}
\subsection{Création de l'arbre syntaxique du programme}
L'arbre syntaxique est consitué de \emph{Term} afin
qu'il puisse être généré directement par l'analyseur syntaxique. 
La méthode getChildList() de la classe Term renvoie la liste des \emph{Term} enfants.


Le code de l'analyser syntaxique prend en entrée un flux Term qui sont tous terminaux. 
L'analyseur syntaxique n'a besoin que des méthodes définie dans l'interface Term pour construire l'arbre.
Voici cette interface : 
{\tiny
\begin{verbatim}

public interface Term {
	  /**
	  * 
	  * @return true if the term is a terminal term
	  */
	  public boolean isTerminal();
	  
	  /**
	  * 
	  * @return Le type du term pour l'analyseur syntaxique.
	  */
	  public String getType();
	
	  /**
	  * 
	  * @return la valeur du term, c'est à dire la chaine de caractère
	  * parsée par l'analyseur lexical.
	  * Si le term n'est pas terminal, il n'a pas de valeur
	  */
	  public String getValue();
	
	  /**
	  * Modifie la valeur du term
	  * @param v la nouvelle valeur du term
	  * @return this
	  */
	  public Term setValue(String v);
	
	  /**
	  * Renvoie la liste des enfants du terme, cette liste n'est pas vide que
	  * si le term n'es pas terminal
	  * @return
	  */
	  public List<Term> getChildList();
	
	  /**
	  * Imprime l'arbre qui est représenté par le term
	  * Indent définit le niveau d'indentation, lorsqu'on imprime
	  * l'arbre entier le point de départ est 0
	  * @param indent
	  */
	  void	printTree(int indent);
	
	
}
\end{verbatim}}


Une fois le code déjà donné dans le chapitre 6 exécuté, l'arbre (le Term) passe dans le treeOrganiser.

La première chose à faire avec l'arbre généré par l'arbre brut est
de mettre tout ce qui est au même niveau de parenthèses au même niveau
dans l'arbre. Pour cela on le parcourt et on fusionne récursivement tous
les $l\_blocks$ et les $l\_lists$ dans leurs parents. Comme les \emph{Term}
gardent en mémoire le type donné à l'analyse syntaxique, les terminaux
récupèrent automatiquement le bon type. 
 
%\begin{figure}
% \centering
% \includegraphics[bb=0 0 428 264]{./restruc.png}
 % restruc.png: 428x264 pixel, 72dpi, 15.10x9.31 cm, bb=0 0 428 264
% \caption{Schema simplifié de la restructuration de l'arbre}
% \label{restruc}
%\end{figure}


Voici la spécification de la seul méthode public du \textit{TreeOrganiser} :
\begin{verbatim}
 /**
  * Contracte l'arbre sortit par l'analyseur syntaxique. 
    Tout les l_list et l_block sont
  * retiré, à la fin il ne reste que des l0 qui sont 
    soit des terminaux ou des listes de l0.
  * @return L'arbre représenté sous la forme d'un lexicalTerm
  */
public LexicalTerm contract()
\end{verbatim}

La structure de donnée LexicalTerm est très similaire à Term, elle implémente l'interface on a rajouté une méthode getLexicalTerm
qui est le type donnée par le parser lexical. Ces types ont été listé dans la partie sur l'analyseur lexical. Une fois 
l'arbre mis en forme et les types des termes réels révélés (pas ceux juste présent pour construire l'arbre), on passe à la traduction.

\section{traduction}
Il y a trois grande famille de traduction, la traduction des définitions de méthode, la traduction des instructions et la traduction des conditions.
La traduction des définitions de méthode est assez facile car on sait qu'un programme en Happy est une liste de méthode. Donc les enfants
de la racine sont les méthodes, l'enfant 0 de l'enfant est le mot clé, l'enfant 1 sont les paramètres et l'enfants 2 est la suite d'instruction.
\begin{verbatim}

/**
	 * Point de départ de l'exploration de l'arbre. 
	 * t est un term qui une liste de term qui est en accord avec la définition
	 * des méthodes ou des fonctions.
	 * Si t n'est pas conforme aux définitions, une erreur de syntaxe est 
  lancée et le programme
	 * se termine
	 * @param t 
	 * @return la Cmethod 
	 */
	private Cmethod analyseMethod(LexicalTerm t) { 

 
\end{verbatim}

Une fois les arguments et le nom de la méthode récupéré il faut commencé à ajouté toutes les instructions à au corps de la méthode.
La méthode privée addBody crée une liste vide qui accueillera toutes les instructions de la méthode. Cette table permet de placer en premier
les instructions qui se trouvent au fond de l'arbre pour gérer correctement les appels imbriqués.

Les instructions dans le langage Happy sont toutes des expressions, c'est-à-dire qu'elle renvoie une valeur. Contrètement dans la traduction
cela ce fait par l'assignation d'une valeur (qui dépend du type d'instruction) variable annonyme qui sera renvoyé à la méthode appelante. Cette méthode
appelante utilisateur la variable si elle en a besoin.

Voici le code de la méthode \textit{addOutput}
{\tiny
\begin{verbatim}

 
/**
 * Ajoute l'instruction output à la liste des instructions, les expréssions
 * imbriquée dans cette instruction seront ajouter avant l'exécution de celle-ci
 * @param child la liste des arguments de l'instruction write, 
   size == 2, l'élément 0 est le term write
 * et le seconde une expression
 * @param table la table des variables local
 * @param body la table contenant les instructions de la méthode 
 * @param counter le compteur des variables annonymes
 * @return la variable qui contient la valeur de retour de l'instruction
 */
private Cvar addOutput(List<LexicalTerm> child, MethodTable table,
   List<Ccmd> body, AnonymousCounter counter) {
  //si l'expression est terminal on peut la traiter tout de suite
  if(child.get(1).isTerminal()) {
    Cvar v1 = null;
    if(WordIdentifier.isRexpr(child.get(1))) {
      Crexpr r = WordIdentifier.getRexpr(child.get(1)); 
      String name = counter.getAnonymousName(); 
      table.addLocal(name);
      v1 = new Cvar(name);				
      body.add(new Cass(v1,r ));
    }
    else if(child.get(1).getLexicalTerm().equals("id")) {
      v1 = new Cvar(child.get(1).getValue());
      table.addLocal(child.get(1).getValue());
    }
    //erreur terminal mais ni une expression ni un id
    else {
      System.out.println("Expected number id or expr");
      System.out.println("in write");
      System.exit(15);
    }
    Clexpr[] expr = {v1};
    Coutput out = new Coutput(expr);
    body.add(out);
    return v1;
  }
  //sinon on explore la commande et on récupère la valeur de l'expression
  else {
    Cvar v1 = exploreCmd(child.get(1), table, body, counter);
    Clexpr[] expr = {v1};
    Coutput out = new Coutput(expr);
    body.add(out);
    return v1;
  }
}
 \end{verbatim}
}
Dans ce code on voit bien les deux cas de figure : les arguments de l'instruction sont des terminaux ou non. 
On voit aussi l'utilisation des variables annonymes grâce à la classe AnonymousCounter qui a comme seul objectif de fournir des noms différents
pour chaque variable annonyme d'une méthode. 

 
Voici un petit de l'exécution de la méthode addOutput :

Considérons le code Happy suivant :
\begin{verbatim}
(write 
  (set a 
    (+ 7 8)
  )
)
sera traduit en:
A_0 = 7 ; //l'expression 7 est assignée à A_0
A_1 = 8 ; //idem pour 8
A_2 = (A_0 + A_1) ; //l'expression (+ A_0 A_1) est assignée à A_2
a = A_2 ; //assignation de l'expression (+ A_0 A_1) à a en passant par A_2
A_3 = a ; //set renvoie la variable qu'il vient d'assigner
write(A_3) ;

\end{verbatim}

Les conditions fonctionne un peu comme l'exploration des commandes, mais ici il faut distinguer deux types de conditions.
Les conditions avec un opérateur logique (and, or, not) et les conditions avex un opérateur de relation. Les première ne peuvent avoir 
comme argument que des conditons tandis que les second ne peuvent avoir comme argument que des expressions. 

Voici un exemple de condition avec un opérateur logique :
{\tiny
\begin{verbatim}


/**
 * Cette fonction renvoie une condition de type and.
 * @param child la liste des Term qui forme la conditions, 
    size == 3 et les éléments 1 et 2 doivent être des conditions
 * @param table la table des variables locales
 * @param body la table contenant la liste des commandes du corps de la méthode
 * @param counter le compteur de variable annonyme
 * @return La condition and.
 */
private Ccond getAnd(List<LexicalTerm> child, MethodTable table, 
  List<Ccmd> body, AnonymousCounter counter) {
  if(child.get(1).isTerminal() || child.get(2).isTerminal()) {
    System.out.println("Syntax error at AND, expected condition");
    System.exit(16);
  }
  return new Cand(getCond(child.get(1), table, body, counter),
      getCond(child.get(2), table, body, counter));
}
 
\end{verbatim}}

Ici encore on voit les appels récursif à getCond (en effet l'appel de getAnd vient uniquement d'un appel à getCond). 
Les conditons avec un opérateur de relation fonctionne exactement comme les instructions. 
L'implémentation avec les expressions imbriquées dans les relations pose un petit soucis. Les expressions sont placé au dessus du if
ou du while et leur résultat sont placés dans une variable annonyme. Cette variable est ensuite utilisée dans la condtion.
Dans un if ceci ne pose pas de problème mais dans un while on risque d'avoir très vite une boucle infinie car les expressions ne
sont plus réévaluée une fois dans le while. (Il faudrait donc aussi rajouté les instructions imbriquée à la fin du while mais l'implémentation
dans sa forme actuelle ne permet pas de le faire facilement). C'est pourquoi le code suivant est vivement déconseillé
{\tiny
\begin{verbatim}
[code qui écris les chiffre de 1 à 10]
 (
        (fun (main) (
                (set i 0)
                (while (<= (set i (+ i 1)) 10) (
                        (write i)

                ))

        ))
)
[La sortie est tout autre => une infinité de 1]

[Voici le code conseillé]
 (
        (fun (main) (
                (set i 1)
                (while (<= i 10) (
                        (write i)
                        (set i (+ i 1))

                ))

        ))
)

[Ce code va effectivement sortir les chiffres de 1 à 10]
\end{verbatim}}

\section{traduction du code en code interne}
Ici nous avons utilisé la méthode Cprog.translate(). Qui nous à poser quelque problème lorsque qu'il y avait dans la traduction en code 
structuré. En effet les erreurs renvoyée ne sont pas des plus claires.

Voici le premier programme de la démo qui fait la somme de 1 à n.
{\tiny
\begin{verbatim}
 


(

[Commentaire]
(fun (main) (
  (set i 0)
  (while (<= i 10)
  (
    (write (sum i))
    (set i (+ i 1))
  ))
))

(fun (sum n) (
  (if (= n 0)
    (return 0)
  )
  (return (+ n (sum (- n 1))))
  ))
)


Traduction en code structuré

main()
{
  i = 0 ;
  A_0 = i ;
  A_1 = 10 ;
  while (i <= A_1)
    {
      A_2 = sum(i) ;
      write(A_2) ;
      A_3 = 1 ;
      A_4 = (i + A_3) ;
      i = A_4 ;
      A_5 = i ;
      A_6 = 0 ;
    }
  A_7 = 0 ;
  A_8 = 0 ;
}


sum(n)
{
  A_0 = 0 ;
  if (n == A_0)
    {
      A_1 = 0 ;
      return (A_1) ;
    }
  A_2 = 0 ;
  A_3 = 1 ;
  A_4 = (n - A_3) ;
  A_5 = sum(A_4) ;
  A_6 = (n + A_5) ;
  return (A_6) ;
  A_7 = 0 ;
}

Traduction en code interne


method main()
{ lab16
  [ lab16 : i#1 := 0 ; go to lab15]
  [ lab15 : A_0#2 := i#1 ; go to lab14]
  [ lab14 : A_1#3 := 10 ; go to lab5]
  [ lab5 : if A_1#3 < i#1 then go to lab3 else go to lab4]
  [ lab3 : A_7#9 := 0 ; go to lab2]
  [ lab2 : A_8#10 := 0 ; go to lab1]
  [ lab4 : skip ; go to lab13]
  [ lab13 : #11 := i#1 ; go to lab12]
  [ lab12 : A_2#4 := sum(#11) ; go to lab11]
  [ lab11 : write(A_2#4) ; go to lab10]
  [ lab10 : A_3#5 := 1 ; go to lab9]
  [ lab9 : A_4#6 := i#1 + A_3#5 ; go to lab8]
  [ lab8 : i#1 := A_4#6 ; go to lab7]
  [ lab7 : A_5#7 := i#1 ; go to lab6]
  [ lab6 : A_6#8 := 0 ; go to lab5]
} lab1 result#0
end of method main.

method sum(n)
{ lab29
  [ lab29 : A_0#2 := 0 ; go to lab28]
  [ lab28 : if n#1 == A_0#2 then go to lab27 else go to lab25]
  [ lab27 : A_1#3 := 0 ; go to lab26]
  [ lab26 : result#0 := A_1#3 ; go to lab17]
  [ lab25 : A_2#4 := 0 ; go to lab24]
  [ lab24 : A_3#5 := 1 ; go to lab23]
  [ lab23 : A_4#6 := n#1 - A_3#5 ; go to lab22]
  [ lab22 : #10 := A_4#6 ; go to lab21]
  [ lab21 : A_5#7 := sum(#10) ; go to lab20]
  [ lab20 : A_6#8 := n#1 + A_5#7 ; go to lab19]
  [ lab19 : result#0 := A_6#8 ; go to lab17]
} lab17 result#0
end of method sum.

Et finalement l'exécution
0
1
3
6
10
15
21
28
36
45
55


\end{verbatim}}
\chapter{Sémantique Opérationelle}


\section{Domaines sémantiques}
\paragraph{Valeurs} Val $\doteq$ Entier + Référence + \{null\} + \{undefined\} + \{Error\}

Une Valeur représente soit un entier soit une réference vers une autre valeur
ou un autre objet. Les cas limite sont aussi compris. Il s'agit de Null pour 
une réference ne pointant vers rien  et de NonInit pour un Int qui n'aurait 
pas été initialisé. \emph{Undefined} représente une variable dont la valeur n'a
pas encore été spécifiée et dont le type n'est donc pas connu. Les champs
des objets fraichement alloués sont de type \emph{undefined}.

\paragraph{Environnement} Env $\doteq \mathbb{X}$ + \{this\} $\rightarrow$ Val

Un environnement est une fonction assurant la correspondence entre une variable ou \emph{this} et la valeur de celle-ci.

\paragraph{Store} Store $\doteq$ Ref $\rightarrow$ $<n,<v_1,v_2,...,v_n> >$	%TODO la flèche comme ca : -|-> 

Store est une fonction. La valeur d'une variable pointant vers un objet est une référence. Plusieurs 
variables peuvent donc avoir des références de même valeur et pointer vers le 
même objet. Le store est une fonction qui permet d'établir une correspondance 
entre la référence et la valeur de l'objet.

\paragraph{Pile} Pile $\doteq$ $<$ Env, Label, X, $<$ Pile $> >$ 

La pile permet au programme de gérer les changements d'environnement qui interviennent
à chaque appel ou retour de fonction. La pile est une pile contenant à chaque étage 
l'environnement courant, le label et la variable de retour. Une pile n'est bien évidemment pas infinie,il est cependant assez difficile de modéliser cette contrainte de manière formelle.

\paragraph{Etats} Etat $\doteq <e,l,s,P,in,out>$\\

$e \in $ Env   \\
$l \in $ Label \\
$s \in $ Store \\
$P \in $ Pile  \\
L'état comprend tout ce qui est nécessaire pour continuer l'exécution du programme, en excluant le
code source de celui-ci, cela veut dire l'environnement, le label de l'instruction courante, le store,
ainsi que la pile d'appel de fonction. 

\paragraph{in} In $\doteq$ Pile $\rightarrow$ Val

La fonction In permet de lire la première valeur de la pile ( donc la dernière valeur entrée par l'utilisateur ).

\paragraph{Out} $\doteq$ Val  $\rightarrow$ Val

La fonction Out permet d'écrire Val au sommet de la pile, Elle renvoie par ailleurs cette même valeur.

\section{Philosophie et choix de design}
	Définir la sémantique opérationnelle ne se limite pas à formaliser des concepts génériques
	comme présenté dans le précédent chapitre. Il faut souvent choisir entre plusieurs alternatives
	et ces choix étendent ou limitent les possibilités du langage.

	Afin de garantir une certaine cohérence dans tout cela, nous avons choisi de tenir une 
	certaine ligne de conduite justifiée ici.

	Tout d'abord deux observations: premièrement le langage \emph{SLIP} n'est pas destiné à 
	être utilisé directement par le programmeur puisque nous allons définir une syntaxe plus 
	agréable dans une seconde partie du projet. Ensuite, dans tout langage il faut choisir 
	un équlibre entre flexibilité et facilité de détection d'erreurs. Par exemple un programme
	typé dynamiquement est plus rapide à écrire mais son exécution est plus difficile à prévoir,
	et donc la garantie que le programme est correct est plus difficile à fournir.

	Nous avons fait le choix d'avoir un langage qui n'essaie pas de deviner ce que le programmeur
	voulait faire, et reporte les erreurs dès que possible. 

	Ainsi, c'est au programmeur de s'assurer qu'il n'y a pas de divisions par zéro. Il nous semble
	plus important que le programme s'arrête en indiquant l'erreur, plutôt que de poursuivre en
	faisant des approximations, en ne prévenant pas le programmeur de la non exactitude des résultats
	obtenus. 

\section{Fonctions sémantiques}
\paragraph{Conditions} B: Cond $\rightarrow$ Env $\rightarrow$ Store $\rightarrow \{true, false, error\}$

$B[$ expr1 cop expr2 $]$ e s\\
$v_1 = V[$expr1$]$ e s\\
$v_2 = V[$expr2$]$ e s\\
$\xi[$cop$]$ $v_1$ $v_2 = \xi_1$ $v_1$ $v_2$ si cop$= '<'$\\
$\xi[$cop$]$ $v_1$ $v_2 = \xi_2$ $v_1$ $v_2$ si cop$= '='$\\
\\
$\xi_1 :$ Val $\rightarrow$ Val $\rightarrow$ $\{true,false,error\}$\\
$\xi_1$ $v_1$ $v_2 = $ true si $v_1 < v_2$ avec $v_1,v_2 \in $ Int\\
$\xi_1$ $v_1$ $v_2 = $ false si $v_1 \geq v_2$ avec $v_1,v_2 \in $ Int\\
$\xi_1$ $v_1$ $v_2 = $ error sinon\\

$\xi_2 :$ Val $\rightarrow$ Val $\rightarrow$ $\{true,false,error\}$\\
$\xi_2$ $v_1$ $v_2$ =  true si $v_1 = v_2$ avec $v_1,v_2 \in $ Int\\
$\xi_2$ $v_1$ $v_2$ =  true si $v_1 = v_2$ avec $v_1,v_2 \in $ Ref\\
$\xi_2$ $v_1$ $v_2$ =  true si $v_1,v_2 \in $ \{null\} \\
$\xi_2$ $v_1$ $v_2$ =  false si $v_1 \neq v_2$ avec $v_1,v_2 \in $ Int\\
$\xi_2$ $v_1$ $v_2$ =  false si $v_1 \neq v_2$ avec $v_1,v_2 \in $ Ref\\
$\xi_2$ $v_1$ $v_2$ =  false si $v_1  \in $ \{null\} et $v_2 \in Ref$\\
$\xi_2$ $v_1$ $v_2$ =  false si $v_1  \in $ Ref et $v_2 \in $ \{null\}\\
$\xi_2$ $v_1$ $v_2$ =  error sinon\\

Cette fonction définit le comportement des conditions. Nous avons choisi une
définition assez stricte ou seul des élèments de même type peuvent être comparés. 
Les comparaisons avec \emph{undefined} renvoie une erreur car on ne sait pas se 
prononcer sur la comparaison d'élément inconnus. 

\paragraph{Designateur} $D$: Des $\rightarrow$ Env $\rightarrow$ Store $\rightarrow$ Val + {Ref,Int} +\{$error$\} \\
$D[x] = x$ si $x \in \mathbb{X}$ \\
$D[x]$ = $error$ sinon \\
$D[x.i] = [ $Env$(x), i ]$ si $ x \in \mathbb{X}$ et $x \in dom($Env$) $ et Env($x$) $\in dom($Store$)$ et si
	$<n,<v_1,v_2,...,v_n> > =$Store($x$) et $ i \in [0,n[$ \\
$D[x.i]$ = $error$ sinon \\
$D[this.i]$ = \{Env$(this), i\}$ si Env($this$) $\in dom($store$)$ et si
	$< n,< v ...>> = $Store($this$) et $ i \in [0,n[$ \\
$D[this.i]$ = $error$ sinon \\
Un désignateur renvoie la variable correspondante, ou une paire référence, indice, selon la forme du désignateur.
le designateur renvoie une erreur si l'on accède à un champ d'un objet non défini, ou que l'indice est en dehors
des limites de l'objet. 

\paragraph{Expressions} $V:$ Expr $\rightarrow$ Env $\rightarrow$ Store $\rightarrow$ Val + \{error\}\\
$V[Val] = Val$\\
$V[Des] = Env[D[$Des$]]$ si $D[$Des$] \in dom($Env$)$ \\
$V[Des] = n $ si $D[$Des$]$ = $\{ref,i\}$ et $i = 0$ et Store($ref$) = $<n,<...>>$ \\
$V[Des] = v_i$ si $D[$Des$]$ = $\{ref,i\}$ et $i \neq 0$ et Store($ref$) = $<n,<v_1,...,v_i,...,v_{n}>$\\
$V[Des] = error $ sinon\\
\\
$V[$Expr1 op Expr2$] = V_+[E_1 E_2] $ si op $= '+'$\\
$V[$Expr1 op Expr2$] = V_-[E_1 E_2] $ si op $= '-'$\\
$V[$Expr1 op Expr2$] = V_*[E_1 E_2] $ si op $= '*'$\\
$V[$Expr1 op Expr2$] = V_/[E_1 E_2] $ si op $= '/'$\\
$V[$Expr1 op Expr2$] = V_\%[E_1 E_2] $ si op $= '\%'$\\
$V[$Expr1 op Expr2$] = error$ sinon\\
\\
$V_+[E_1 E_2] = V[E_1] + V[E_2]$ si $V[E_1]$ et $V[E_2] \in$ int \\
$V_+[E_1 E_2] = error$ sinon.\\
\\
$V_-[E_1 E_2] = V[E_1] - V[E_2]$ si $V[E_1]$ et $V[E_2] \in$ int \\
$V_-[E_1 E_2] = error$ sinon.\\
\\
$V_*[E_1 E_2] = V[E_1] * V[E_2]$ si $V[E_1]$ et $V[E_2] \in$ int \\
$V_*[E_1 E_2] = error$ sinon.\\
\\
$V_/[E_1 E_2] = V[E_1] / V[E_2]$ si $V[E_1]$ et $V[E_2] \in$ int, et que  $V[E_2] \neq 0$\\
$V_/[E_1 E_2] = error$ sinon.\\
\\
$V_\%[E_1 E_2] = V[E_1] \% V[E_2]$ si $V[E_1]$ et $V[E_2] \in$ int et que $V[E_2] \neq 0$\\
$V_\%[E_1 E_2] = error$ sinon.\\
\\

Il est bon de noter qu'aucun dépassement des bornes n'est possible, et ce car l'ensemble des valeurs comprend tout l'ensemble des naturels (Cela sera implémenté en java grâce à la classe BigInteger, mais il n'est pas encore temps d'y penser ;-) )

\paragraph{Affectations} $A :$ Ass $\rightarrow$  Env $\rightarrow$ Store $\rightarrow$ {Env,Store} + \{$error$\}  \\
$A[[des := expr]] e s = A1[[des]][[expr]] e s$\\
$A[[x := new/i ]] e s = A2[[x]][[i]] e s$\\
Sinon $error$
\newline
$A1 : $Des$ \rightarrow $Expr$ \rightarrow $Env$ \rightarrow $Store$ \rightarrow$ {NewEnv,NewStore} + \{$error$\} \\ 
Si $Env[D[$Des$]] \in {undefined}$ et $V[$Expr$] \neq error$ alors NewEnv = Env $\bigoplus$ $D[$Des$]$ $\rightarrow$ $V[$Exp$]$ et NewStore = Store\\
\newline
Si $Env[D[$Des$]] \in$ Int et $V[$Expr$] \in$ Int alors NewEnv = Env $\bigoplus$ $D[$Des$]$ $\rightarrow$ $V[$Exp$]$ et NewStore = Store\\
\newline
Si $D[$Des$] = \{ref, i\} \in$ \{Ref, Int\} et $<n,<...,v_i,...> = Store[ref]$ et $i \neq 0$ et $v_i \in {undef}$ et $V[$Expr$] \neq error$ alors NewEnv = Env et NewStore = Store $\bigoplus ref \rightarrow <n,<...v_{i-1},V[$Expr$],v_{i+1}...>>$\\ 
\newline
Si $D[$Des$] = \{ref, i\} \in$ \{Ref, Int\} et $<n,<...,v_i,...> = Store[ref]$ et $i \neq 0$ et $v_i \in$Int et $V[$Expr$] \in $Int alors NewEnv = Env et NewStore = Store $\bigoplus ref \rightarrow <n,<...v_{i-1},V[$Expr$],v_{i+1}...>>$\\ 
\newline
Si $D[$Des$] = \{ref, i\} \in$ \{Ref, Int\} et $<n,<...,v_i,...> = Store[ref]$ et $i \neq 0$ et $v_i \in$Ref + \{$null$\} et $V[$Expr$] \in $Ref + \{$null$\} alors NewEnv = Env et NewStore = Store $\bigoplus ref \rightarrow <n,<...v_{i-1},V[$Expr$],v_{i+1}...>>$\\ 
\newline
$A2 : $X$ \rightarrow $Int$ \rightarrow $Env$ \rightarrow $Store$ \rightarrow$ {NewEnv,NewStore} + \{$error$\} \\ 
Si $Env[$x$] \in Ref+{undefined}$ et $i > 0$ et $r \in Ref$ et $r \not\in dom(Store)$  alors  
	NewEnv = Env $\bigoplus x \rightarrow r$ et NewStore = Store $\bigoplus r \rightarrow <i,<v_0,...,v_j,...,v_{i}>> | \forall(j) v_j = undefined$\\
	$error$ sinon\\

L'opérateur $\bigoplus$ désignant la surimpression fonctionnelle, c'est à dire que  pour une fonction $f : x \rightarrow y $ , la surimpression par un couple  $ a \rightarrow b $, signifie que $f(x)=y$ pour $x \neq a$ et $f(x)=b$ pour $x = a$\\

L'affectation force le respect des types. Il n'est pas possible d'assigner une référence à une variable qui était 
précédement un entier, et inversément. Comme il n'y a pas d'instruction permettant de savoir quel est le type d'une variable il 
nous a semblé plus raisonnable d'imposer le respect des types afin que le programmeur puisse savoir à tout moment
de quel type est chaque variable. 

Il n'est pas non plus possible de modifier la taille d'un objet, et lorsque ceux-ci sont instanciés, tous leur champs sauf
celui à zéro sont mis à \emph{undefined}.  

\paragraph{In/Out}
L'input est représenté par une pile de valeurs entières sur laquelle le programme lit ses entrées. 
In = $< val \in $Int$, <$In$>>$\\
$I : in \rightarrow Env \rightarrow In \rightarrow {In,Env} + {error}$\\
Si $in = read x$ et $x \in X$ et $ Env(x) \in Int + {undefined} $ et $In = < val, <In2>>$ alors $NewIn = In2$ et $NewEnv = Env \bigoplus x \rightarrow val $\\
$error$ sinon\\

De manière similaire l'output est représenté par une pile de valeur entière sur laquelle le programme lit ses entrées. 
Out = $< val \in $char*$ <$Out$>>$\\
$O : out \rightarrow Env \rightarrow  Out \rightarrow NewOut $\\
Si $out = write x$ et $x \in dom(Env)$ et $Env(x) \in Int$ NewOut = $<tochar(Env(x)),<Out>>$\\
Si $out = write x$ et $x \in dom(Env)$ et $Env(x) \in Ref$ NewOut = $<$"ref:"$+tochar(Env(x)),<Out>>$\\
Si $out = write x$ et $x \in dom(Env)$ et $Env(x) \in {undefined}$ NewOut = $<$"undefined"$,<Out>>$\\

La fonction \emph{tochar(x)} prend un \emph{int} ou une référence en paramètre et la convertis en une séquence de caractère. 
Dans le cas d'une référence, ce qui est imprimé dépend de l'implémentation et n'a pas de garantie d'être
constant entre les plateformes ou exécutions successives (de la même façon qu'en pointeur en C). 

\section{Relations de transition}
\paragraph{Conditions}
$<e,l,s,P,in,out> \rightarrow$ $l$ if cond then $l1$ else $l2$  $\rightarrow <e,l',s,P,in,out>$\\
$l'$ = $l1$ si B[cond] = $true$\\
$l'$ = $l2$ si B[cond] = $false$\\
si B[cond]  = $error$ alors le programme s'arrête en affichant un message d'erreur.

\paragraph{Affectations}
$<e,l,s,P,in,out> \rightarrow$ $l$ ass $l'$ $\rightarrow <e',l',s',P,in,out>$\\
$(e',s')$ = A[[ass]]$ e\: s$\\
si A[[ass]]$ e s$ = $error$ alors le programme s'arrète en affichant un message d'erreur.

\paragraph{Entrée/Sortie} 
$<e,l,s,P,in,out> \rightarrow$ $l$ read x $l'$  $\rightarrow <e',l',s,P,in',out>$\\
$in',e' = $ I[[read x]] $e\: in$\\
si I[[read x]] $e in$ = $error$ alors le programme s'arrète en affichant un message d'erreur.

$<e,l,s,P,in,out> \rightarrow$ $l$ write x $l'$  $\rightarrow <e,l',s,P,in,out'>$\\
$out' = $O[[write x]] $e\: out$
si O[[write x]] = $error$ alors le programme s'arrète en affichant un message d'erreur. 
e' et s' sont les valeurs renvoiée par la fonction d'affectation.

\paragraph{appels de méthodes statiques}
$<e,l,s,P,in,out> \rightarrow x = m(x_1,...,x_n) \rightarrow <e',l',s,P',in,out>$\\
$e'$ = $x_i \rightarrow V[x_i]$ soit l'ensemble des paramètres associés à leur valeur d'appel. \\
$l'$ est le label de la première instruction de la méthode m.\\
$P'$ = $ < l e x <P>>$ Soit la pile $P$ à laquelle on ajoute le label $l$,l'environnement $e$, et la variable de retour $x$\\
La méthode est connue et choisie grâce à la réference m

\paragraph{appels de méthodes dynamique}
$<e,l,s,P,in,out> \rightarrow x = y.m(x_1,...,x_n) \rightarrow <e',l',s,P',in,out>$\\
$e'$ = $x_i \rightarrow V[x_i] \bigoplus this \rightarrow Env(y)$ soit l'ensemble des paramètres associés à leur valeur d'appel auquel
on ajoute la définition du this.\\
$l'$ est le label de la première instruction de la méthode \emph{m/i} ou i est l'entier le plus grand qui soit plus petit ou égal à
$n$ défini par $<n,<...>> = Store(Env(y))$\\
$P'$ = $ < l e x <P>>$ Soit la pile $P$ à laquelle on ajoute le label $l$,l'environnement $e$, et la variable de retour $x$\\
La méthode est connue et choisie grâce à la réference m

\paragraph{appels vers super}
$<e,l,s,P,in,out> \rightarrow x = super.m(x_1,...,x_n) \rightarrow <e',l',s,P',in,out>$\\
$e'$ = $x_i \rightarrow V[x_i] \bigoplus this \rightarrow Env(y)$ soit l'ensemble des paramètres associés à leur valeur d'appel auquel
on ajoute la définition du this.\\
$l'$ est le label de la première instruction de la méthode \emph{m/i} ou i est l'entier le plus grand qui strictement plus petit que 
$n$ défini par $<n,<...>> = Store(Env)$\\
$P'$ = $ < l e x <P>>$ Soit la pile $P$ à laquelle on ajoute le label $l$,l'environnement $e$, et la variable de retour $x$

\paragraph{retour de fonction}
$<e,l,s,P,in,out> \rightarrow x_r \rightarrow <e',l',s,P',in,out>$\\
$P = <l',e'',x,P'>$\\
$e'= e''\bigoplus x \rightarrow e(x_r)$\\
C'est a dire qu'on remplace l'environnement par celui qui était sur la pile, dans lequel
on change la variable de retour par la valeur de retour, et l'on change le label
par le label de retour. 

\paragraph{démarrage du programme}
le programme démarre au début de la fonction 'main' sans paramètres, avec une pile,un environnement et un store vide. C'est à dire : \\
$<e,l,s,P,in,out> \rightarrow main()$ \\
$e = \{\}$\\
$P = \{\}$\\
$s = \{\}$





%10
\chapter{Mode d'emploi du compilateur}
La version exécutable du compilateur et interpréteur Happy se trouve dans le jar exécutable \textit{happy.jar}.
Il prend 2 arguments obligatoire : la grammaire au format BNF et le programme en langage Happy. On peut rajouter \textit{check} à la fin pour vérifier la grammaire en plus.
Pour faire fonctionner les programmes Happy, il faut passer la grammaire \textit{temp.bnf}. Pour vérifier n'importe quelle grammaire WP, il faut indiquer la grammaire
un fichier programme bidon et enfin check.
exemple : \begin{verbatim}
    java -jar happy.jar temp.bnf programme1.happy 
    java -jar happy.jar temp.bnf programme1.happy check
          \end{verbatim}

L'interpréteur sort les informations suivantes dans cet ordre, si le check de la grammaire est demandé, le résultat des tests et la table de précédence.
Ensuite l'arbre sortit par le parser syntaxique après sa restructuration, ensuite la traduction de l'arbre en SLIP et puis (mais il semble que
les retour à la ligne ne soit pas conforme ce qui donne des résultats étranges à la sortie dans le shell) la représentation en code interne SLIP. Et finalement
la sortie du programme interprèté. 


\section{Les erreurs}
Lorsqu'une erreur arrive dans n'importe quelle partie du compilateur l'erreur est affichée (avec plus ou moins de précision) et il s'arrête ensuite.
\section{Les erreurs du vérificateur}
Le vérificateur de grammaire WP évalue les critères les uns après les autres et s'arrète dès qu'un critère n'est pas respecté, en indiquant à la console
l'endroit où cela coince de manière assez explicite.  

\section{Les erreurs du parser lexical}
  Il ne vérifie que deux choses. A la fin de l'analyse si il n'y a pas le même nombre de paranthèse ouvrantes que fermantes, l'erreur \textit{Unexpected end}.
  Il est aussi capable de vérifier si à tout moment il y a trop de paranthèse fermantes, le messages est dès lors très explicite : \textit{Too much ) }.
\section{Les erreurs de l'analyseur syntaxique}
  L'analyseur syntaxique signale trois erreurs. Lorsque deux termes ne peuvent se retrouver côte à côte, lorsqu'il n'arrive pas à trouver une règle pour réduire.


Et lorsqu'il a lu tout les caractères, si il reste plus d'un élément dans la pile une erreur est aussi renvoyé.
\section{Les erreurs du traducteur}
    Les erreurs que renvoient le traducteur sont d'ordre sémantique. Le programme est valide syntaxiquement mais comme notre grammaire permet beaucoup de chose,
Il faut remettre les choses en places avec le traducteur. Les erreurs sont du type : quelquechose est attendu dans ce type d'expr et là c'est pas le cas. 
exemple : textit{Expected id after a set}


\section{Les erreurs de l'interpréteur}
  L'interpréteur reporte les erreurs d'exécutions telles que la division par 0, l'appel à méthode inconnue etc...
  Les erreurs sont clairement identifié dans le code interne structurée et reporté jusqu'au dessus de la pile d'exécution. Cela permet de retracer l'erreur
sans trop de difficulté. 
Voici un exemple : 
\begin{verbatim}
Error divide by 0
	 in Cexpr a#1 / b#2
	 at Ass A_0#3 := a#1 / b#2
	 at CmdStmt [ lab9 : A_0#3 := a#1 / b#2 ; go to lab8]
	 at divide/-1
	 at Call divide
	 at CmdStmt [ lab2 : A_0#3 := divide(#4, #5) ; go to lab1]
	 at main/-1
\end{verbatim}

\section{Exemples de programmes}
	Voici un programme triant une liste avec l'algorithme quicksort. 
	{\small \begin{verbatimtab}
	[Ce programme montre un exemple
	de quick sort avec une générère avec
	une fonction "pseudo alétoire"]
(
	(fun (# val) (
		(set n (new 2))
		(set n.1 val)
		(set n.2 null)
		(return n)
	)) 
	
	[@param N une liste 
	 @post imprime la liste	]

	(fun (Print N)(
		(if (! (= N null)) ( 
			(write N.1)
			(Print N.2)
		))
	))
	(method.2 (print) (Print this))
	
	[Ajoute la valeur val à la liste N2]

	(fun (#> Val N2) 
		(return (-> (# Val) N2)) 
	)
	
	[ajoute N2 à la suite du noeud N1]

	(fun (-> N1 N2) (
		(set N1.2 N2)
		(return N1)
	))
	
	[Met la liste L2 après L1]
	(fun (Join L1 L2) (
		(if (= L1 null) (return L2) )
		(set R L1)
		(set Temp L1.2)
		(while (!(= Temp null)) (
			
			(set L1 L1.2)
			(set Temp L1.2)
		))
		(set L1.2 L2)
		(return R)
	))
	
	[Renvoie une copie de la liste L triée]
	(fun (Qsort L) (
		(if (= L null) (return null))
		(if (= L.2 null) (return L))
			
		(set Pivot L.1)
		(set L L.2)
		(set L1 null)
		(set L2 null)
		(while (!(= L null)) (
		      (if (<= L.1 Pivot) 
				(set L1 (#> L.1 L1))
				(set L2 (#> L.1 L2))
			)
			(set L L.2)
		))
		(return (Join (Qsort L1) (Join (# Pivot) (Qsort L2 ))))
	))

	[Lit un nombre en entrée et trie une liste de cette longueur]
	(fun (main) (
		(set i (read))
		(set List (# 2))
		(set seed 8)
		(while (> i 0) (
			(set seed (random 429496726 seed))
			(set List (#> seed List)) 
			(set i (- i 1))
		))
		(set List (Qsort List))
		(List.print)
	))

	[Fonction pseudo random
	 @param max : le nombre max
	 @param seed : la graine de départ]
		 
	(fun (random max seed)  
		(return (% (+ 1013904223 (* 1664525 seed)) max))
	) 
)
	\end{verbatimtab}}
\subsection{Traduction en langage abstrait structuré}
{\small \begin{verbatimtab}
#(val)
{
  A_0 = new/2 ;
  n = A_0 ;
  A_1 = n ;
  n.1 = val ;
  A_2 = n.1 ;
  n.2 = null ;
  A_3 = n.2 ;
  return (n) ;
  A_4 = 0 ;
}

Print(N)
{
  A_0 = null ;
  if (!(N == A_0))
    {
      A_1 = N.1 ;
      write(A_1) ;
      A_2 = N.2 ;
      A_3 = Print(A_2) ;
      A_4 = 0 ;
    }
  A_5 = 0 ;
  A_6 = 0 ;
}

print/2()
{
  A_0 = this ;
  A_1 = Print(A_0) ;
}

#>(Val, N2)
{
  A_0 = #(Val) ;
  A_1 = ->(A_0, N2) ;
  return (A_1) ;
}

->(N1, N2)
{
  N1.2 = N2 ;
  A_0 = N1.2 ;
  return (N1) ;
  A_1 = 0 ;
}

Join(L1, L2)
{
  A_0 = null ;
  if (L1 == A_0)
    {
      return (L2) ;
    }
  A_1 = 0 ;
  R = L1 ;
  A_2 = R ;
  Temp = L1.2 ;
  A_3 = Temp ;
  A_4 = null ;
  while (!(Temp == A_4))
    {
      L1 = L1.2 ;
      A_5 = L1 ;
      Temp = L1.2 ;
      A_6 = Temp ;
      A_7 = 0 ;
    }
  A_8 = 0 ;
  L1.2 = L2 ;
  A_9 = L1.2 ;
  return (R) ;
  A_10 = 0 ;
}

Qsort(L)
{
  A_0 = null ;
  if (L == A_0)
    {
      A_1 = null ;
      return (A_1) ;
    }
  A_2 = 0 ;
  A_3 = L.2 ;
  A_4 = null ;
  if (A_3 == A_4)
    {
      return (L) ;
    }
  A_5 = 0 ;
  Pivot = L.1 ;
  A_6 = Pivot ;
  L = L.2 ;
  A_7 = L ;
  L1 = null ;
  A_8 = L1 ;
  L2 = null ;
  A_9 = L2 ;
  A_10 = null ;
  while (!(L == A_10))
    {
      A_11 = L.1 ;
      if (A_11 <= Pivot)
        {
          A_12 = L.1 ;
          A_13 = #>(A_12, L1) ;
          L1 = A_13 ;
          A_14 = L1 ;
        }
      else
        {
          A_15 = L.1 ;
          A_16 = #>(A_15, L2) ;
          L2 = A_16 ;
          A_17 = L2 ;
        }
      A_18 = 0 ;
      L = L.2 ;
      A_19 = L ;
      A_20 = 0 ;
    }
  A_21 = 0 ;
  A_22 = Qsort(L1) ;
  A_23 = #(Pivot) ;
  A_24 = Qsort(L2) ;
  A_25 = Join(A_23, A_24) ;
  A_26 = Join(A_22, A_25) ;
  return (A_26) ;
  A_27 = 0 ;
}

main()
{
  read(A_0) ;
  i = A_0 ;
  A_1 = i ;
  A_2 = 2 ;
  A_3 = #(A_2) ;
  List = A_3 ;
  A_4 = List ;
  seed = 8 ;
  A_5 = seed ;
  A_6 = 0 ;
  while (i > A_6)
    {
      A_7 = 429496726 ;
      A_8 = random(A_7, seed) ;
      seed = A_8 ;
      A_9 = seed ;
      A_10 = #>(seed, List) ;
      List = A_10 ;
      A_11 = List ;
      A_12 = 1 ;
      A_13 = (i - A_12) ;
      i = A_13 ;
      A_14 = i ;
      A_15 = 0 ;
    }
  A_16 = 0 ;
  A_17 = Qsort(List) ;
  List = A_17 ;
  A_18 = List ;
  A_19 = List.print() ;
  A_20 = 0 ;
}

random(max, seed)
{
  A_0 = 1013904223 ;
  A_1 = 1664525 ;
  A_2 = (A_1 * seed) ;
  A_3 = (A_0 + A_2) ;
  A_4 = (A_3 % max) ;
  return (A_4) ;
}
\end{verbatimtab}}
\subsection{Code Interne}
{\tiny \begin{verbatimtab}
method #(val)
{ lab10
  [ lab10 : A_0#3 := new/2 ; go to lab9]
  [ lab9 : n#2 := A_0#3 ; go to lab8]
  [ lab8 : A_1#4 := n#2 ; go to lab7]
  [ lab7 : n#2.1 := val#1 ; go to lab6]
  [ lab6 : A_2#6 := n#2.1 ; go to lab5]
  [ lab5 : n#2.2 := null ; go to lab4]
  [ lab4 : A_3#8 := n#2.2 ; go to lab3]
  [ lab3 : result#0 := n#2 ; go to lab1]
} lab1 result#0
end of method #.

method Print(N)
{ lab21
  [ lab21 : A_0#2 := null ; go to lab20]
  [ lab20 : if N#1 == A_0#2 then go to lab13 else go to lab19]
  [ lab13 : A_5#7 := 0 ; go to lab12]
  [ lab12 : A_6#8 := 0 ; go to lab11]
  [ lab19 : A_1#3 := N#1.1 ; go to lab18]
  [ lab18 : write(A_1#3) ; go to lab17]
  [ lab17 : A_2#4 := N#1.2 ; go to lab16]
  [ lab16 : #9 := A_2#4 ; go to lab15]
  [ lab15 : A_3#5 := Print(#9) ; go to lab14]
  [ lab14 : A_4#6 := 0 ; go to lab13]
} lab11 result#0
end of method Print.

method print/2()
{ lab25
  [ lab25 : A_0#1 := this ; go to lab24]
  [ lab24 : #3 := A_0#1 ; go to lab23]
  [ lab23 : A_1#2 := Print(#3) ; go to lab22]
} lab22 result#0
end of method print.

method #>(Val, N2)
{ lab32
  [ lab32 : #5 := Val#1 ; go to lab31]
  [ lab31 : A_0#3 := #(#5) ; go to lab30]
  [ lab30 : #5 := A_0#3 ; go to lab29]
  [ lab29 : #6 := N2#2 ; go to lab28]
  [ lab28 : A_1#4 := ->(#5, #6) ; go to lab27]
  [ lab27 : result#0 := A_1#4 ; go to lab26]
} lab26 result#0
end of method #>.

method ->(N1, N2)
{ lab37
  [ lab37 : N1#1.2 := N2#2 ; go to lab36]
  [ lab36 : A_0#4 := N1#1.2 ; go to lab35]
  [ lab35 : result#0 := N1#1 ; go to lab33]
} lab33 result#0
end of method ->.

method Join(L1, L2)
{ lab59
  [ lab59 : A_0#3 := null ; go to lab58]
  [ lab58 : if L1#1 == A_0#3 then go to lab57 else go to lab56]
  [ lab57 : result#0 := L2#2 ; go to lab38]
  [ lab56 : A_1#4 := 0 ; go to lab55]
  [ lab55 : R#5 := L1#1 ; go to lab54]
  [ lab54 : A_2#6 := R#5 ; go to lab53]
  [ lab53 : Temp#7 := L1#1.2 ; go to lab52]
  [ lab52 : A_3#8 := Temp#7 ; go to lab51]
  [ lab51 : A_4#9 := null ; go to lab45]
  [ lab45 : if Temp#7 == A_4#9 then go to lab43 else go to lab44]
  [ lab43 : A_8#13 := 0 ; go to lab42]
  [ lab42 : L1#1.2 := L2#2 ; go to lab41]
  [ lab41 : A_9#15 := L1#1.2 ; go to lab40]
  [ lab40 : result#0 := R#5 ; go to lab38]
  [ lab44 : skip ; go to lab50]
  [ lab50 : L1#1 := L1#1.2 ; go to lab49]
  [ lab49 : A_5#10 := L1#1 ; go to lab48]
  [ lab48 : Temp#7 := L1#1.2 ; go to lab47]
  [ lab47 : A_6#11 := Temp#7 ; go to lab46]
  [ lab46 : A_7#12 := 0 ; go to lab45]
} lab38 result#0
end of method Join.

method Qsort(L)
{ lab114
  [ lab114 : A_0#2 := null ; go to lab113]
  [ lab113 : if L#1 == A_0#2 then go to lab112 else go to lab110]
  [ lab112 : A_1#3 := null ; go to lab111]
  [ lab111 : result#0 := A_1#3 ; go to lab60]
  [ lab110 : A_2#4 := 0 ; go to lab109]
  [ lab109 : A_3#5 := L#1.2 ; go to lab108]
  [ lab108 : A_4#6 := null ; go to lab107]
  [ lab107 : if A_3#5 == A_4#6 then go to lab106 else go to lab105]
  [ lab106 : result#0 := L#1 ; go to lab60]
  [ lab105 : A_5#7 := 0 ; go to lab104]
  [ lab104 : Pivot#8 := L#1.1 ; go to lab103]
  [ lab103 : A_6#9 := Pivot#8 ; go to lab102]
  [ lab102 : L#1 := L#1.2 ; go to lab101]
  [ lab101 : A_7#10 := L#1 ; go to lab100]
  [ lab100 : L1#11 := null ; go to lab99]
  [ lab99 : A_8#12 := L1#11 ; go to lab98]
  [ lab98 : L2#13 := null ; go to lab97]
  [ lab97 : A_9#14 := L2#13 ; go to lab96]
  [ lab96 : A_10#15 := null ; go to lab77]
  [ lab77 : if L#1 == A_10#15 then go to lab75 else go to lab76]
  [ lab75 : A_21#26 := 0 ; go to lab74]
  [ lab74 : #33 := L1#11 ; go to lab73]
  [ lab73 : A_22#27 := Qsort(#33) ; go to lab72]
  [ lab72 : #33 := Pivot#8 ; go to lab71]
  [ lab71 : A_23#28 := #(#33) ; go to lab70]
  [ lab70 : #33 := L2#13 ; go to lab69]
  [ lab69 : A_24#29 := Qsort(#33) ; go to lab68]
  [ lab68 : #33 := A_23#28 ; go to lab67]
  [ lab67 : #34 := A_24#29 ; go to lab66]
  [ lab66 : A_25#30 := Join(#33, #34) ; go to lab65]
  [ lab65 : #33 := A_22#27 ; go to lab64]
  [ lab64 : #34 := A_25#30 ; go to lab63]
  [ lab63 : A_26#31 := Join(#33, #34) ; go to lab62]
  [ lab62 : result#0 := A_26#31 ; go to lab60]
  [ lab76 : skip ; go to lab95]
  [ lab95 : A_11#16 := L#1.1 ; go to lab94]
  [ lab94 : if Pivot#8 < A_11#16 then go to lab87 else go to lab93]
  [ lab87 : A_15#20 := L#1.1 ; go to lab86]
  [ lab86 : #33 := A_15#20 ; go to lab85]
  [ lab85 : #34 := L2#13 ; go to lab84]
  [ lab84 : A_16#21 := #>(#33, #34) ; go to lab83]
  [ lab83 : L2#13 := A_16#21 ; go to lab82]
  [ lab82 : A_17#22 := L2#13 ; go to lab81]
  [ lab81 : A_18#23 := 0 ; go to lab80]
  [ lab80 : L#1 := L#1.2 ; go to lab79]
  [ lab79 : A_19#24 := L#1 ; go to lab78]
  [ lab78 : A_20#25 := 0 ; go to lab77]
  [ lab93 : A_12#17 := L#1.1 ; go to lab92]
  [ lab92 : #33 := A_12#17 ; go to lab91]
  [ lab91 : #34 := L1#11 ; go to lab90]
  [ lab90 : A_13#18 := #>(#33, #34) ; go to lab89]
  [ lab89 : L1#11 := A_13#18 ; go to lab88]
  [ lab88 : A_14#19 := L1#11 ; go to lab81]
} lab60 result#0
end of method Qsort.

method main()
{ lab151
  [ lab151 : read(A_0#2) ; go to lab150]
  [ lab150 : i#1 := A_0#2 ; go to lab149]
  [ lab149 : A_1#3 := i#1 ; go to lab148]
  [ lab148 : A_2#5 := 2 ; go to lab147]
  [ lab147 : #25 := A_2#5 ; go to lab146]
  [ lab146 : A_3#6 := #(#25) ; go to lab145]
  [ lab145 : List#4 := A_3#6 ; go to lab144]
  [ lab144 : A_4#7 := List#4 ; go to lab143]
  [ lab143 : seed#8 := 8 ; go to lab142]
  [ lab142 : A_5#9 := seed#8 ; go to lab141]
  [ lab141 : A_6#10 := 0 ; go to lab124]
  [ lab124 : if A_6#10 < i#1 then go to lab123 else go to lab122]
  [ lab123 : skip ; go to lab140]
  [ lab140 : A_7#11 := 429496726 ; go to lab139]
  [ lab139 : #25 := A_7#11 ; go to lab138]
  [ lab138 : #26 := seed#8 ; go to lab137]
  [ lab137 : A_8#12 := random(#25, #26) ; go to lab136]
  [ lab136 : seed#8 := A_8#12 ; go to lab135]
  [ lab135 : A_9#13 := seed#8 ; go to lab134]
  [ lab134 : #25 := seed#8 ; go to lab133]
  [ lab133 : #26 := List#4 ; go to lab132]
  [ lab132 : A_10#14 := #>(#25, #26) ; go to lab131]
  [ lab131 : List#4 := A_10#14 ; go to lab130]
  [ lab130 : A_11#15 := List#4 ; go to lab129]
  [ lab129 : A_12#16 := 1 ; go to lab128]
  [ lab128 : A_13#17 := i#1 - A_12#16 ; go to lab127]
  [ lab127 : i#1 := A_13#17 ; go to lab126]
  [ lab126 : A_14#18 := i#1 ; go to lab125]
  [ lab125 : A_15#19 := 0 ; go to lab124]
  [ lab122 : A_16#20 := 0 ; go to lab121]
  [ lab121 : #25 := List#4 ; go to lab120]
  [ lab120 : A_17#21 := Qsort(#25) ; go to lab119]
  [ lab119 : List#4 := A_17#21 ; go to lab118]
  [ lab118 : A_18#22 := List#4 ; go to lab117]
  [ lab117 : A_19#23 := List#4.print() ; go to lab116]
  [ lab116 : A_20#24 := 0 ; go to lab115]
} lab115 result#0
end of method main.

method random(max, seed)
{ lab158
  [ lab158 : A_0#3 := 1013904223 ; go to lab157]
  [ lab157 : A_1#4 := 1664525 ; go to lab156]
  [ lab156 : A_2#5 := A_1#4 * seed#2 ; go to lab155]
  [ lab155 : A_3#6 := A_0#3 + A_2#5 ; go to lab154]
  [ lab154 : A_4#7 := A_3#6 % max#1 ; go to lab153]
  [ lab153 : result#0 := A_4#7 ; go to lab152]
} lab152 result#0
end of method random.
\end{verbatimtab}}
\subsection{Exemple d'exécution}
{\small \begin{verbatimtab}
10
2
15870471
32357778
65091998
168226971
172331438
234389844
298521463
307405643
325022690
385540813
\end{verbatimtab}}

\section{Conclusion}
	Ce projet fut pour nous l'occasion de découvrir l'envers
	du décor des compilateurs et interpréteurs, programmes que nous
	utilisons quotidiennement sans jamais savoir comment ils fonctionnent réellement.

	Nous avons aussi vu comment concevoir un langage de programmation, compétence
	indispensable pour tout ingénieur en informatique. 

	Ce fut aussi l'occasion de découvrir le lisp que nous ne connaissions
	que de réputation. Nous poursuirons très certainement l'apprentissage
	de ce language très élégant, et certains membres du groupes se sont
	même mis l'idée en tête d'en écrire un interpréteur plus complet. 

	Nous trouvons cependant dommage qu'il ne nous ai pas été demmandé plus
	tôt de réaliser l'analyseur syntaxique. En effet, réaliser celui ci
	n'est pas très compliqué et permet de trouver des erreurs dans la syntaxe,
	que nos vérificateurs n'avaient pas trouvés. 

	Nous trouvons aussi dommage le manque de ressources écrites sur la syntaxe
	WP. Tout n'était pas clair et parfois l'assistant n'était pas en mesure de
	nous aider. Ce projet était un défi très intéressant il est dommage que 
	les ressources ne soient pas à sa hauteur. 

	Enfin nous ne pouvons ommettre que la charge de travail dépasse largement
	Le cadre d'un cours à 5ects. Ce qui semble d'ailleurs être un problème 
	général des cours d'INFO. Il ne faut pas oublier qu'avec 6 cours et autant
	de groupes différents nous perdons beaucoup de temps à nous organiser et
	qu'il n'est pas facile de trouver des plages horaires qui permettent de
	satisfaire la présence au cours de tout le monde. 

	Nous somme en tout cas satisfait de notre travail et de ce projet et 
	espérons que vous l'êtes aussi. :) 



	



\end{document}
