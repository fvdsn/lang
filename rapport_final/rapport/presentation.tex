\chapter{Présentation de Happy:-)}
\section{Introduction}
La langage happy, appelé ainsi parce qu'il est très permissif au niveau des caractères permit dans les identifiants comporte de nombreuses autres particularités. Tout d'abord, le langage ressemble très fort au LISP où tout est atom ou liste.
Mais contrairement au LISP, notre langage est impératif et orienté objet. Un programme est une liste de méthode, une méthode est une liste dont le premier élément est le mot réservé fun le second une list d'argument et le dernier une liste de commande. Même principe pour le while et le if. Dans ce langage tout est fonction, dans le sens que toute instruction renvoie une valeur, y compris les if et le while qui renvoient 0, une fonction qui n'a pas d'instruction return renvoie null, le write renvoie la valeur qu'elle vient d'imprimer etc...



Les conditions aurait aussi renvoyé une valeur si le langage SLIP dans lequel notre langage est traduit le permettait. 

\section{Grammaire de départ}
Cette grammaire est la grammaire exhaustive du langage si on rajoute que les id peuvent être formé de tout les caractères UTF-16 sauf des caractères réservés et qu'ils ne doivent pas être égale à un mot réservé.

\begin{verbatim}
<Program> ::= ( <Prog_list> )
<Prog_list> ::=  <Meth_or_fun> | <Prog_list> <Meth_or_fun>	
<Meth_or_fun> ::=  <Method> | <Function>

<Function> := ( fun ( <Arglist> ) <Instr_list> ) 
<Arglist> ::=  id | id <Arglist>
<Method> ::= ( <Method_int> ( <Arglist> ) <Instr_list> )
 
<Instr_list> ::=  <Instr> | ( <Instr_list_np> ) 
<Instr_list_np> ::=  <Instr> | <Instr> <Instr_list_np>
<Instr> ::=  <Conditional> | <While_block> | <Call> | ( return <Expr> ) 

<Conditional> ::= ( if <Cond> <instr_list> <instr_list>  )  
<Conditional> ::= ( if <Cond> <instr_list> )
<While_block> ::= ( while  <Cond> <Instr_list>  ) 
<call> ::=  <User_call> | <Method_call> | <Builtin_call> 
<Builtin_call> ::=  <Assignment> | <Read_call> | <Write_call> | <Arithmetic_call>
<Assignment> ::= ( set id <Expr> ) 
<Assignment> ::= ( set <Id_int> <Expr> ) | ( set <This_int> <Expr> )
<Read_call>  ::= ( read )
<Write_call> ::= ( write <Expr> )
<Arithmetic_call> ::=  <Binary> | <Unary>
<Binary> ::= ( <Bin_id> <Expr> <Expr>  )
<Unary>  ::= ( <Un_id> <Expr>  )
<User_call>  ::= ( id  <Expr_list_np> ) | ( id ) 
<Expr_list_np> ::=  <Expr> | <Expr> <Expr_list_np>
<Method_call>  ::= ( <Id_id> <Expr_list_np>  ) |  ( <Id_id> )   
<Method_call>  ::= ( <Super_id>  <Expr_list_np>  ) | ( <Super_id>  )
<Method_call>  ::= ( <This_id>  <Expr_list_np>  ) | ( <This_id> )
<Expr> ::= number | null | true | false | this | id 
<Expr> ::= <Id_int> | <This_int> | <Instr>
<Cond> ::= <Rel> | ( <Log_bin_op> <Cond> <Cond> )
<Cond> ::= ( <Log_un_op> <Cond> <Cond> )
<Rel>  ::= ( <Rel_op> <Expr> <Expr> ) 
<Rel_op> ::= <= | >= | > |  < | =
<Log_bin_op> ::= and | or
<Log_un_op> ::= !
<Bin_id> ::= + | - | * | / | %
<Un_id> ::= neg
<Id_int> ::= id . number
<This_int> ::= this . number
<Super_id> ::= super . id
<This_id> := this . id 
<Id_id> ::= id . id
<Method_int> ::= method . number
\end{verbatim}
Cette grammaire n'est pas wp, mais elle exprime très bien ce qui est syntaxiquement correcte dans ce que nous avons réellement implémenté. 

Voici quelque bout de code permis par cette grammaire :

\begin{verbatim}
[Le while s'écrit comme ceci while (la condition) (les instruction a répèter)
  ici la condition est 3 * i <= 9
On remarque aussi que sur ce bout de code on peut écrire 
la valeur de retour de set qui sera ici i_initial + 1 ou i_final ]
(while (<= (* 3  i) 9) (write (set i (+ i 1)))) 
  
[Condition ici si i != 9 ]
[Ensuite si vrai on exécute la première list d'instruction sinon la seconde]
(if (! (= i 9)) (return true) (return false))
[Ceci est équivalent à sauf que dans le cas deux on voit 
  clairement les listes d'instructions]
(if (! (= i 9)) ((return true)) ((return false)))

(fun (++ i) (return (+ i 1)))

(set i (++ i))

(set a (new 2))
\end{verbatim}

\section{Exemple de programme complet}
Le premier programme imprime juste l'entier lu à la console.

\begin{verbatim}
(
  (fun (main) 
    (write (read))
  )
)
\end{verbatim}

Le programme suivant fait la somme de 1 à n pour 10 n
\begin{verbatim}
(
  [Programme fait la somme de 1 à n pour n qui va de 0 à 10]
  (fun (main) (
    (set i 0)
    (while (<= i 10)
    (
      (write (sum i))
      (set i (+ i 1))
    ))
  ))


  (fun (sum n) (
    (if (= n 0)
      (return 0)
    )
    (return (+ n (sum (- n 1))))
  ))
)

\end{verbatim}



Le dernier programme fait la somme des éléments d'une pile et utilise la POO
\begin{verbatim}
(
  (fun (main) (
    (set s ( >> 4 ( >> 3 (>> 2 (# 1)))))
    (write (s.@))
    (write (s.->))
    (set s (>> 5 s))
    (Print s)
    (write (sum s))
  ))
  [crée une nouvelle pile avec a comme élément au sommet]
  (fun (# a) (
    (set b (new 2))
    (b.@= a)
    (b.->= null)
    (return b)
  ))

  [Push sur la stack]
  [a : l'élément à mettre sur la stack]
  [s : la stack]
  (fun (>> a s) (
    (set n (new 2))
    (n.->= s)
    (n.@= a)
    (return n)
  )) 

  (fun (Print N) (
    (if (! (= N null)) (
      (write (N.@))
      (Print (N.->))
    )
    (write 0)
    )
  ))

  (fun (sum N) (
    (if (! (= N null)) 
      (return (+ (N.@) (sum (N.->))))
      (return 0)
    )
     
  ))
  [Accesseur pour l'élément contenu dans le noeud]
  (method.2 (@) (return this.1))
  ((method.2 (@= a) (set this.1 a))
  [Accesseur pour next]
  (method.2 (->) (return this.2))
  (method.2 (->= a) (set this.2 a))
)
\end{verbatim}




 