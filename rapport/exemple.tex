\section{Exemple de programmation HAPPY}


\subsection{Exemple1}
Un programme complet.
{\tiny 
\begin{verbatim}
(fun (:-D) (write 42))
(fun (** a b) (
  (set cpt b)
  (set res 0)
  (while (> cpt 0) 
    (
      (set r (* res b))
      (set cpt (- cpt 1))	
    )
  )
  (return R)
))
(fun (main) (
  (HelloWorld)
  (set B (** 4 2))
  (return (write B))
))

(method (4) (-> level) (return this.level))

\end{verbatim}
}
\subsection{Exemple2}
Exemple de code intéressant.
{\tiny 
\begin{verbatim}

(method.2 (next) (return this.1))
(method.2 (val)  (return this.2))
  [Creation d'un noeud de linked list]
(fun (N x) ((set n (new 2)) (set n.2 x) (set n.1 null) (return n)) )
  [Ajoute un noeud a une linked list] 
(fun (-> Node NextNode) ( (set Node.1 NextNode) (return Node)) )

(fun (main) (
  (set A (->(N 1) (->(N 2) (->(N 3) (N 4)))))
  (while A (
    (write (A.val))
    (set A (A.next))
  ))
  (return 0)
))
\end{verbatim}
}
\subsection{Exemple3}

Voici un exemple de programmation Happy :
{\tiny
\begin{verbatim}

(fun (Adder A B)(
  (set C (+ A B))
  (return C)
)

(fun (exposant A B)(
  (set result A)
  (set I 0)
  (while (I<B) 
    (set result (* result A))
    (set I (+ I 1))
  )
  (return result)
  )
)

(method(3) (Moore A B)(
  (return (A*B*1.5))
  )
)

(fun (main) (
  (set A 5)
  (set B 10)
  (set C (Adder A B))
  (set D (Exposant C B))
  (write (moore A B))
  )
)

\end{verbatim}
}
Dans cet exemple on montre différente manière de traiter des nombres grâce à des opérations relativement simple, tel que l'addition et l'exposant. Cet exemple permet d'illustrer la syntaxe de manière concrète.
