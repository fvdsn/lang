\section*{Syntaxe concrètes}

\subsection{Syntaxe initiale}
Nous avons définit un premier lieu cette syntaxe : 

\begin{verbatim}
<Reserved_words>    ::= set | if | while | write | fun | method 
      | null | true | false | read | new | + | - | * | / 
      | % | '|' | & | ! | < | > | <= | >= | neg | this 
      | return | skip
<Reserved_caracter> ::= <o_brackets> | <c_brackets> | .
<digit> ::= 0..9 
<pnumber> ::=  <digit> | <number> <digit> 
<number>  ::= - <pnumber> | <pnumber>
<caracter> ::= a..z | A..Z | ! | @ | # | $ | % | ^ | & | * 
		| { | } '|' | | _ | , | ? | - | + | = | / 
		| \\ | > | < | : | ; | ~ | \ | " 
<sp> ::= lambda |   | NewLine | Tab | <sp> <sp>
<esp> ::=   <sp>
<Id> ::= <caracter> | <Id> <caracter> | <Id> <digit>   
<meth_or_fun> ::= <method> | <function>
<Program> ::=  <meth_or_fun> | <Program> <sp>  <meth_or_fun> <esp>
<function> ::= (<sp> fun <sp> ( <sp> <Name> <esp> <arglist> <sp>) 
		<sp> <instr_list> <sp>) 
<arglist> ::= <Id> | <Id> <esp> <arglist>
<methode> ::= (<sp> method <sp> (<sp> <Number> <sp>) <sp> 
	      ( <sp> <Name> <esp> <arglist> <sp> ) <instr_list>  <sp> )
<instr_list> ::= <instr> | ( <sp> <instr_list_np> <sp> ) 
<instr_list_np> ::= <instr> | <instr> <esp> <instr_list_np>
<instr> ::= <conditional> | <while_block> | <call> 
			  | ( <sp> return <esp> <expr> <sp> ) 
			  | (<sp> skip <sp>)
<conditional> ::= (<sp> if <esp> <expr> <esp> <instr_list>
		  <esp> <instr_list> <sp> ) 
<while_block ::= ( <sp> while <esp> <expr> <esp> <instr_list> <sp> ) 
<call> ::= <function_call> | <method_call> | <builtin_call> 
<builtin_call> ::= <assignment> | <read_call> | <write_call> | <arithmetic>
<assignment> ::= ( <sp> set <esp> <left_Id> <esp> <expr> <sp> ) 
<read_call>  ::= ( <sp> read <sp> ) 
<write_call> ::= ( <sp> write <esp> <expr> )
<arithmetic> ::= <binary> | <unary>
<binary ::= ( <sp> <bin_id> <esp> <expr> <esp> <expr> <sp> )
<bin_id> ::=  + | - | * | / | % | '|' | & | < | > | <= | >=
<unary>  ::=  ( <sp> <un_id> <esp> <expr> <sp> )
<un_id>  ::=  ! | neg | new
<left_Id> ::= <Id> | <Id> . <expr> | this .  <expr>
<user_call>  ::= ( <sp> <Id> <esp> <expr_list_np> <sp> )
<expr_list_np> ::= lambda | <expr> | <expr> <esp> <expr_list_np>
<method_call>  ::= ( <sp> <method_id> <esp> <expr_list_np> <sp> )
<method_id> ::= <Id> . <Id> | super . <Id> 
<expr> ::= <call> | <left_Id> | <number> | null | true | false | this  
\end{verbatim}


Vu que cette syntaxe n'utilise que les parenthèses et le points comme caractère réservés, nous pouvons utiliser n'importe caractère pour définir
un identifiant (pour peut qu'il ne commence pas par un chiffre). Nous pouvons donc donner des noms tel que $:-D$ ou encore $;\}$ voir $:->$. Nous avons donc choisit d'appeler notre langage le langage Happy. 

Les commentaires sont entre [ ] : [mon commentaire]

Les noms des fonctions \textit{builtin} du langage sont des mots réservés afin d'éviter d'écrire des fonctions utilisateur qui aurait le même nom
Ces fonctions sont l'assignement (set), le new, les opérations arithmétique et logique.

Voici quelque exemple de programme écrit en happy : 
\begin{verbatim}
(fun (:-D) (write 42))
(fun (** a b) (
	(set cpt b)
	(set res 0)
	(while (> cpt 0) 
		(
			(set r (* res b))
			(set cpt (- cpt 1))	
		)
	)
	(return R)
))

(fun (main) (
	(HelloWorld)
	(set B (** 4 2))
	(write B)
	(set A (new 5))
	(-> A 2)
))

(method (4) (-> level) (return this.level))

\end{verbatim}

\subsection{Syntaxe de l'analyseur Lexical}

Notre analyseur lexical nous permet de parser notre programme et d'en ressortir une liste de jetons ordonnés concret. La syntaxe qui sera analysée ne sera bien évidemment pas aussi fine que la syntaxe de base définie dans la section précedente ( en effet, quel serait l'objectif de ressortir des jetons contenant chacun un et un seul caractère ! ). Les jetons contiendront donc des unités concrètes tel que les identifieurs, les nombes, les nombres négatifs, les parenthèse, etc etc..

Voici donc la syntaxe interpretée par l'analyseur syntaxique :

\subsection{Syntaxe initiale}
Nous avons définit un premier lieu cette syntaxe : 

\begin{verbatim}
<Reserved_words>    ::= set | if | while | write | fun | method 
      | null | true | false | read | new | + | - | * | / 
      | % | '|' | & | ! | < | > | <= | >= | neg | this 
      | return | skip
<meth_or_fun> ::= <method> | <function>
<Program> ::=  <meth_or_fun> | <Program>  <meth_or_fun>
<function> ::= (fun ( Id <arglist> ) <instr_list> ) 
<arglist> ::= Id | Id <arglist>
<methode> ::= (method ( Number ) (Id <arglist> ) <instr_list> )
<instr_list> ::= <instr> | ( <instr_list_np> ) 
<instr_list_np> ::= <instr> | <instr> <instr_list_np>
<instr> ::= <conditional> | <while_block> | <call> 
			  | ( return <expr> ) 
			  | ( skip )
<conditional> ::= (if <expr> <instr_list> <instr_list> ) 
<while_block ::= ( while <expr> <instr_list>) 
<call> ::= <function_call> | <method_call> | <builtin_call> 
<builtin_call> ::= <assignment> | <read_call> | <write_call> | <arithmetic>
<assignment> ::= ( set <left_Id> <expr> ) 
<read_call>  ::= ( read ) 
<write_call> ::= ( write <expr> )
<arithmetic> ::= <binary> | <unary>
<binary ::= (<bin_id> <expr> <expr>)
<bin_id> ::=  + | - | * | / | % | '|' | & | < | > | <= | >=
<unary>  ::=  (<un_id> <expr>)
<un_id>  ::=  ! | neg | new
<left_Id> ::= Id | Id . <expr> | this .  <expr>
<user_call>  ::= ( <Id> <expr_list_np> )
<expr_list_np> ::= lambda | <expr> | <expr> <expr_list_np>
<method_call>  ::= ( <method_id> <expr_list_np>)
<method_id> ::= <Id> . <Id> | super . <Id> 
<expr> ::= <call> | <left_Id> | <number> | null | true | false | this  
\end{verbatim}


\subsection{Syntaxe WP de l'analyseur grammatical}
