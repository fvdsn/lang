\section*{Syntaxe concrètes}

Notre langage s'inspire fortement du lisp, langage qui nous attirait par sa simplicité
et son élégance. Toutes les structures sont constituées de liste d'élément séparés par
des espaces, clos par des parenthèses. 

La syntaxe BNF suivante aurait pu être nettement plus simple, cependant nous avons choisi
de rendre explicites certains éléments de la syntaxe. Ainsi nous différentions le Nom ( <name> )
de la fonction de ses arguments, même si dans certains cas ce sont des entités syntaxiques identiques. 

Les appels de fonction sont de forme \emph{(Nom arg1 arg2 ...)}. A chaque opérateur du langage 
correspond une fonction, par exemple \emph{(+ 1 2)} Cependant nous avons donné un statut spécial
de <builtin\_call> à celles-ci afin d'empècher l'utilisateur de les redéfinir, et l'on espère,
de simplifier l'implémentation du futur interpréteur en lui permettant de repérer directement
les fonctions internes. 

Read et write sont eux aussi implémentés en tant que fonction, la fonction write renvoyant
l'argument donné en plus de l'imprimer sur la sortie standard, permettant de les intégrer
à des expressions normales. 

Cette syntaxe a l'avantage d'avoir peu de caractères réservés $\{ (,),.,[,]\}$, et peu
de restrictions sur les identifiants (ceux ci ne peuvent commencer par un chiffre). Il
est donc facile de créer de nouveaux opérateurs, parfois amusant comme des smileys , ce
qui a donné le nom de notre langage.  

Notre langage différe cependant du lisp dans le fait qu'il n'est pas composé d'une seule
expression, mais d'une suite de fonctions et méthodes, et que celles-ci sont composées de
liste d'instructions. les instructions peuvent ètre des expressions dont la valeur est 
ignorée. Seul if, while, skip et return ne sont pas des expressions.
Le langage ainsi défini est donc impératif, mais garde un petit coté fonctionnel. 

Les objets sont implémentés comme des listes de longueur fixe qui peuvent être crées dynamiquement
grâce à la fonction new.

Les commentaires sont entre brackets et leur contenu est remplacé par un espace par l'analyseur lexical. 

L'ensemble des caractères acceptables par le programme est constitué d'un sous-ensemble de l'ASCII,
constitué de l'union de <caracter> et de <reserved\_caracter>. Les caractères Tab et NewLine sont aussi
reconnu mais leur valeur peut dépendre de la plateforme. 

\subsection{Syntaxe initiale}
Voici la syntaxe en notation BNF.
\begin{verbatim}
<Reserved_words>    ::= set | if | while | write | fun | method 
      | null | true | false | read | this 
<bin_id> ::=  + | - | * | / | % | '|' | & | < | > | <= | >=
<un_id>  ::=  ! | neg | new
<Reserved_caracter> ::=  ( | ) | . | [ | ]
<digit> ::= 0..9 
<pnumber> ::=  <digit> | <number> <digit> 
<number>  ::= - <pnumber> | <pnumber>
<caracter> ::= a..z | A..Z | ! | @ | # | $ | % | ^ | & | * 
		| { | } '|' | | _ | , | ? | - | + | = | / 
		| \\ | > | < | : | ; | ~ | \ | " 
<sp> ::= lambda |   | NewLine | Tab | <sp> <sp>
<esp> ::=   <sp>
<Id> ::= <caracter> | <Id> <caracter> | <Id> <digit>   
<meth_or_fun> ::= <methode> | <function>
<program> ::=  <meth_or_fun> | <program> <sp>  <meth_or_fun> 
<function> ::= (<sp> fun <sp> ( <sp> <Name> <esp> <arglist> <sp>) 
		<sp> <instr_list> <sp>) 
<arglist> ::= <Id> | <Id> <esp> <arglist>
<methode> ::= (<sp> method . <Number> <sp> 
	      ( <sp> <Name> <esp> <arglist> <sp> ) <instr_list>  <sp> )
<instr_list> ::= <instr> | ( <sp> <instr_list_np> <sp> ) 
<instr_list_np> ::= <instr> | <instr> <esp> <instr_list_np>
<instr> ::= <conditional> | <while_block> | <call> 
			  | ( <sp> return <esp> <expr> <sp> ) 
			  | (<sp> skip <sp>)
<conditional> ::= (<sp> if <esp> <expr> <esp> <instr_list>
		  <esp> <instr_list> <sp> ) 
<while_block ::= ( <sp> while <esp> <expr> <esp> <instr_list> <sp> ) 
<call> ::= <function_call> | <method_call> | <builtin_call> 
<builtin_call> ::= <assignment> | <read_call> | <write_call> | <arithmetic>
<assignment> ::= ( <sp> set <esp> <left_Id> <esp> <expr> <sp> ) 
<read_call>  ::= ( <sp> read <sp> ) 
<write_call> ::= ( <sp> write <esp> <expr> )
<arithmetic> ::= <binary> | <unary>
<binary ::= ( <sp> <bin_id> <esp> <expr> <esp> <expr> <sp> )
<unary>  ::=  ( <sp> <un_id> <esp> <expr> <sp> ) | return | skip
<left_Id> ::= <Id> | <Id> . <expr> | this . <expr>
<function_call>  ::= ( <sp> <Id> <esp> <expr_list_np> <sp> )
<expr_list_np> ::= lambda | <expr> | <expr> <esp> <expr_list_np>
<method_call>  ::= ( <sp> <method_id> <esp> <expr_list_np> <sp> )
<method_id>    ::= <Id> . <Id> | super . <Id> 
<expr>         ::= <call> | <left_Id> | <number> | null | true | false | this  
\end{verbatim}

\subsection{Syntaxe de l'analyseur Lexical}

Notre analyseur lexical nous permet de parser notre programme et d'en ressortir une liste de lexèmes ordonnés concret. 
La syntaxe qui sera analysée ne sera bien évidemment pas aussi fine que la syntaxe de base définie dans la section précedente 
( en effet, quel serait l'objectif de ressortir des jetons contenant chacun un et un seul caractère ! ). 
Les lexèmes contiendront donc des unités concrètes tel que les identifieurs, les nombes, les nombres négatifs, les parenthèse, etc etc..

Voici donc la syntaxe interpretée par l'analyseur syntaxique :

\begin{verbatim}
<Reserved_words>    ::= set | if | while | write | fun | method 
      | null | true | false | read | new | return | skip
<meth_or_fun> ::= <method> | <function>
<Program> ::=  <meth_or_fun> | <Program>  <meth_or_fun>
<function> ::= (fun ( Id <arglist> ) <instr_list> ) 
<arglist> ::= Id | Id <arglist>
<methode> ::= (method ( Number ) (Id <arglist> ) <instr_list> )
<instr_list> ::= <instr> | ( <instr_list_np> ) 
<instr_list_np> ::= <instr> | <instr> <instr_list_np>
<instr> ::= <conditional> | <while_block> | <call> 
			  | ( return <expr> ) 
			  | ( skip )
<conditional> ::= (if <expr> <instr_list> <instr_list> ) 
<while_block ::= ( while <expr> <instr_list>) 
<call> ::= <function_call> | <method_call> | <builtin_call> 
<builtin_call> ::= <assignment> | <read_call> | <write_call> 
                | <arithmetic>
<assignment> ::= ( set <left_Id> <expr> ) 
<read_call>  ::= ( read ) 
<write_call> ::= ( write <expr> )
<arithmetic> ::= <binary> | <unary>
<binary ::= (<bin_id> <expr> <expr>)
<bin_id> ::=  + | - | * | / | % | '|' | & | < | > | <= | >=
<unary>  ::=  (<un_id> <expr>)
<un_id>  ::=  ! | neg | new
<left_Id> ::= Id | Id . <expr> | this .  <expr>
<user_call>  ::= ( <Id> <expr_list_np> )
<expr_list_np> ::= lambda | <expr> | <expr> <expr_list_np>
<method_call>  ::= ( <method_id> <expr_list_np>)
<method_id> ::= <Id> . <Id> | super . <Id> 
<expr> ::= <call> | <left_Id> | <number> | null | true | false | this  
\end{verbatim}


\subsection{Syntaxe WP de l'analyseur grammatical}
La Syntaxe de l'analyse grammaticale est de plus haut niveau car elle n'est définie que sur ce qui
est renvoyé par l'analyseur syntaxique. Quelques adaptations ont été nécessaires afin de rendre
notre syntaxe Weak Priority. 

Premièrement toutes les parenthèses se trouvant à l'intérieur d'une rêgle ont du être sorties en
tant que nouvelles règles afin de respecter la précédence entre les parenthèses et les espaces.  

Pour élíminer les terminaux vides, nous avons dédoublé les règles: une pour le cas vide, et l'autre
pour le cas alternatif. Seul le <sp> nous posait problème (<sp> représente un espace / indentation facultatif)
Nous avons décidé de les remplacer par des <esp> (indentation obligatoire). Remplacer les <sp> par des <esp> au
niveau de l'analyse est trivial. Il nous aurait beaucoup plu de retirer entièrement les espaces, mais ceux-ci
sont important pour délimiter les points. Il aurait été possible d'insérer des délimiteurs fictifs autour
des règles avec point, mais cela semblait moins facile à implémenter au niveau de l'analyseur syntaxique. 

Comme la grande partie des règles sont contenues par des parenthèses il n'y a pas eu beaucoup de problème de suffixes
ou d'indéterminisme. Les rares problèmes de suffixes ont pu ètre résolus en remontant les règles problématiques
à l'intérieur de règles avec parenthèses. 

Voici donc la syntaxe WP. 
\begin{verbatim}
<number>	::=  <number> | pnumber		
<meth_or_fun> 	::=  <methode> | <function>
<Program> 	::=  <meth_or_fun> | <Program> esp <meth_or_fun>

<function> 	::= ( esp fun esp <arglist> esp <instr_list> esp ) 
<methode> 	::= ( esp method . pnumber esp <arglist> esp <instr_list> esp ) 

<arglist>	::= ( esp <arglist_np> esp )
<arglist_np> 	::=  id | id esp <arglist_np>

<instr_list> 	::=  <instr> | ( esp <instr_list_np> esp ) 
<instr_list_np> ::=  <instr> | <instr> <instr_list_np>

<instr> 	::=  <conditional> | <while_block> | <call> 
           | ( esp return esp <expr> esp) 
           | ( esp skip esp )
<conditional> 	::= ( esp if esp <expr> esp <instr_list>esp<instr_list> esp ) 
<while_block> 	::= ( esp while esp <expr> esp <instr_list> esp ) 
<call> 		::=  <user_call> | <method_call> | <builtin_call> 
<builtin_call> 	::=  <assignment> | <read_call> | <write_call> | <arithmetic>

<assignment> 	::= ( esp set esp id esp <expr> esp ) 
               | ( esp set esp id . <expr> esp <expr> ) 
               | ( esp set esp this . <expr> esp <expr> esp )
<read_call>  	::= ( esp read esp ) 
<write_call> 	::= ( esp write esp <expr> esp )
<arithmetic> 	::=  <binary> | <unary>
<binary> 	::= ( esp bin_id esp<expr> esp <expr> esp  )
<unary>  	::= ( esp un_id esp <expr> esp )
<user_call>  	::= ( esp id esp  <expr_list_np> esp  ) | ( esp id esp ) 

<method_call>  	::= ( esp id.id esp <expr_list_np> esp ) 
                 | ( esp super.id esp <expr_list_np> esp ) 

<expr_list_np> 	::=  <expr> | <expr> esp <expr_list_np>
<expr> 		::=  <call>  | <number> | null | true | false | this | id | id.<expr> 
           | this.<expr>
\end{verbatim}
