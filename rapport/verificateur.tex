\section{Vérificateur de grammaire}
	\subsection{BNF Parser}
	Le point de départ du vérificateur de grammaire est le parser BNF. Ce parser permet d'avoir sous une forme utilisable que nous détaillerons 
plus loin, les règles contenue dans le fichier. Le loader charge le fichier et le lit ligne par ligne.

Loader
\begin{verbatim*}
	/**
	 * initialize the rules loader, the file should be a valid BNF File
	 * that mean every rules look like that
	 * nonTerminal ::= rules1 | rules2 | .. | rulesN
	 * nonTerminal should be written like that : <id>
	 * Terminal should be written like that : 'term'
	 * special caracter  ' \ could be in a terminal symbole precede by a \
	 * 
	 * exemple : 
	 * <E> ::= <T> 
	 * <E> ::= '\\' <T> 
	 * <E> ::= <T> '+' <E>
	 * <T> ::= <F>
	 * <T> ::= <F> '*' <T> 
	 * <F> ::= '\''
	 *  
	 * @param file the path of the file
	 * @throws FileNotFoundException
	 * @throws IOException
	 * 
	 * the rules can be extract with the method getRules()
	 */
\end{verbatim*}

Avec chaque ligne le loader instancie une rule qu'il place dans une liste. La classe rule parse la ligne et sépare d'un coté le nom et de l'autre
une orListe qui représente une liste de règles \textit{ou}. Chaque éléments de la orList est une catList c'est à dire une liste de \textit{Term} qui forme une règles.

	\subsection{Architecture}
	Le vérificateur de grammaire est implémenté comme une série de test
	statique correspondant aux conditions de Weak Priority.
	
	Tous ces tests prennent en paramètre la grammaire (une liste de "Rules")

	Un test global - \emph{CheckAll} - se charge de tester l'ensemble de ces tests.


	\subsection{Test de symbole vide (\emph{CheckLambda})}
		Ce test s'assure qu'aucune règle n'a de symbole vide (lambda).
		Ce test parcourt simplement l'arbre de la définition grammaire et
		vérifie l'absence de terminaux lambda.
	\subsection{Test de conflit de préférence (\emph{CheckPrecedence)}}
		Ce test vérifie qu'il n'y ait pas de conflit entre les précédence
		de symbole. Le seul conflit autorisé est entre '<.' et '=.' qui donne
		la relation de précédence '<.='.

		Pour faire cela, le test construit deux ensemble pour chaque non-terminal :
		First et Last. First(A) contient l'ensemble des terminaux et non terminaux 
		pouvant se trouver sur l'extrème gauche de A. Last(A) contient l'ensemble
		des terminaux et non terminaux pouvant se trouver sur l'extrème droide de A. 

		Pour calculer Tous les First on utilise l'algorithme suivant :
		\begin{enumerate}
			\item Pour tout non terminal A de la grammaire : On identifie les règles correspondantes.
			Pour chacune d'entre elles on identifie si elles commencent par un terminal, et si
			c'est le cas on les ajoute à First(A).
			\item Pour tout non terminal A de la grammaire : On identifie les règles correspondantes,
			Pour chacune d'entre elles, on identifie si elles commencent par un non terminal B.
			Si c'est le cas on ajoute B et First(B) à First(A). 
			\item on répète l'étape précédente tant que les First changent.
		\end{enumerate}
		L'algorithme pour trouver les Last est identique à ceci près qu'on examine les cotés droit des règles.
		
		Une fois que l'on a ces deux ensembles et que l'on sait qu'il n'y a pas de symboles vide dans
		la grammaire on  peut calculer la table de précedence, grace à l'algorithme suivant :
		\begin{enumerate}
			\item On initialise tous les éléments de la table à 'Nothing'
			\item On identifie toutes les parties droites, et pour chacune d'entre elles on identifie tous les
			symboles cotes à cotes XY.
			\item On ajoute X$.=$Y dans la table.
			\item Si X non terminal : Pour tout symbole s dans Last(X), Pour tout terminal t dans First(Y)
				on met s$.>$t dans la table. Si Y est non-terminal on met s$.>$Y dans la table.
			\item Si Y non terminal : Pour tout symbole s dans First(Y), on met X $<.$ s dans la table.
			\item Si à un moment on met une relation dans la table là ou précédemment il n'y avait pas 'Nothing',
				on a un conflit. Les conflits impliquant $<.$ , =. et $<.=$ sont résolus en insérant $<.=$ dans
				la table. Les autres conflits sont marqués en tant que conflits et reportés. 
			\item Si il y a des conflits non résolus dans la table alors le Test échoue, la grammaire n'est pas valide
			Weak Precedence.
		\end{enumerate}
	\subsection{Test de Suffixe}
		Ce test vérifie que les relations de suffixe sont respectées. C'est à dire :
		Pour tout couple de règles Z $->$ B, X $->$ AYB, où A représente une suite de symbole, Y un symbole unique,
		et B et une suite non vide symboles, on vérifie que !(Y .= X) , !(Y $<.$ X) et !(Y $<.=$ X).
		
		Pour ce faire nous utilisons la méthode suivante : 
		Pour chaque règle Z $->$ B, pour toute autre règle X $->$ Y on regarde si on peut matcher B comme suffixe de Y.
		Dans l'affirmative, on identifie le symbole juste avant le suffixe. Nous pouvons a ce moment vérifier la
		condition décrite expliquée précédemment. Si elle n'est pas vérifiée. le Test échoue, la grammaire n'est pas valide.


\begin{verbatim}
	/**
	 * 
	 * @param rules The rules list
	 * @param table The precedence table
	 * @return a list of tuple (triple) that contains the two conflicting rules and the suffix the cause the conflict
	 */
	public static List<RulesTuple> checkSuffix(List<Rule> rules, Hashtable<Term,Hashtable<Term,String>> table) {
\end{verbatim}
