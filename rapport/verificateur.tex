\section{Vérificateur de grammaire}
	\subsection{BNF Parser}
	Le point de départ du vérificateur de grammaire est le parser BNF. Ce parser permet d'avoir sous une forme utilisable que nous détaillerons 
plus loin, les règles contenue dans le fichier. Le loader charge le fichier et le lit ligne par ligne.

Loader
\begin{verbatim*}
	/**
	 * initialize the rules loader, the file should be a valid BNF File
	 * that mean every rules look like that
	 * nonTerminal ::= rules1 | rules2 | .. | rulesN
	 * nonTerminal should be written like that : <id>
	 * Terminal should be written like that : 'term'
	 * special caracter  ' \ could be in a terminal symbole precede by a \
	 * 
	 * exemple : 
	 * <E> ::= <T> 
	 * <E> ::= '\\' <T> 
	 * <E> ::= <T> '+' <E>
	 * <T> ::= <F>
	 * <T> ::= <F> '*' <T> 
	 * <F> ::= '\''
	 *  
	 * @param file the path of the file
	 * @throws FileNotFoundException
	 * @throws IOException
	 * 
	 * the rules can be extract with the method getRules()
	 */
\end{verbatim*}

Avec chaque ligne le loader instancie une rule qu'il place dans une liste. La classe rule parse la ligne et sépare d'un coté le nom et de l'autre
une orListe qui représente une liste de règles \textit{ou}. Chaque éléments de la orList est une catList c'est à dire une liste de \textit{Term} qui forme une règles.

	\subsection{Architecture}
	Le vérificateur de grammaire est implémenté comme une série de test
	statique correspondant aux conditions de Weak Priority.
	
	Tous ces tests prennent en paramètre la grammaire (une liste de "Rules")

	Un test global se charge de tester l'ensemble de ces tests.



	Le vérificateur de suffixe ne crée pas un arbre de suffix. Il place juste toutes les règles dans une liste et compare chaque règle à cette liste
	Si un suffix est trouvé les critères suivants sont testés et si il existe un problème, le suffix qui est en conflit et les deux règles sont placer dans la liste des conflits

\begin{verbatim}
 /**
	 * 
	 * @param rules The rules list
	 * @param table The precedence table
	 * @return a list of tuple (triple) that contains the two conflicting rules and the suffix the cause the conflict
	 */
	public static List<RulesTuple> checkSuffix(List<Rule> rules, Hashtable<Term,Hashtable<Term,String>> table) {
\end{verbatim}
