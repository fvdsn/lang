\documentclass[a4paper,fleqn,11pt]{article}

% Chargement d'extensions
\usepackage[latin1]{inputenc}
\usepackage[T1]{fontenc}
\usepackage[french]{babel}
\usepackage{listings}

% Modification des marges
\setlength{\oddsidemargin}{0.5cm}
\setlength{\textwidth}{15cm}
\setlength{\textheight}{23.5cm}
\setlength{\topmargin}{-2cm}
\setlength{\headheight}{1cm}

% Modification du style
\setlength{\parindent}{0pt}
\setlength{\parskip}{10pt}

% Style des listes
\lstset{frame=tblr,basicstyle=\sf\scriptsize,columns=fixed,aboveskip=5mm,belowskip=2mm,frameround=tttt,framesep=10pt,xleftmargin=10pt,tabsize=4}

% D�but du document
\begin{document}
\thispagestyle{empty}

% Num�ro du groupe
\hfill Groupe 6

% Nom du langage
{\Huge \sl \bfseries HAPPY-)}


% Exemple qui calcule la factorielle
\vspace{5mm}
\begin{lstlisting}
(fun (main) (write (fact (read))))
(fun (fact x) (
	(if (= x 0) (return 1)	(return (* x (fact (- x 1)))))	
))
\end{lstlisting}

% Description
\vspace{0.1cm}\hspace{1cm}
\begin{minipage}{14cm}
\sf La syntaxe du HAPPY-) est simple et �pur�e � l'image du LISP dont il langage est inspir�. Tout est une fonction (ou une m�thode) dans ce langage sauf le while et le if : le write renvoie l'objet qu'il vient d'�crire, le set la valeur qu'on vient d'assign�e (ex (write (set a 5))). Mais rien n'oblige le programmeur � retourner quelquechose dans les fonctions qu'il �crit lui m�me. Les caract�res autoris�s pour d�finir les identifiants d�passent les simples caract�re alphanum�rique, on peut donc d�finir de nouveaux op�rateurs pour nos donn�es tel que -> ou ::= ou encore <<<< voir :-> (la fonction happy). Mais puisqu'un exemple vaut mieux qu'un long discours, voici un programme travaillant avec une pile.  
\end{minipage}


% Un autre petit programme au choix
\vspace{0.1cm}
\begin{lstlisting}
(fun (main) (
	(set :: (newStack 1))
	(>>> 7 (>>> 6 (>>> 5 (>>> 4 (>>> 3 (>>> 2 ::))))))
	(write (sum ::))
))
(fun (sum stack) (
	(set sum 0)
	(while ((set val (<<< stack))) ((set sum (+ sum val)))
	(return sum)
))
(fun (newStack x) (
	(set stack (new 1)) (set node (newNode)) (node:= x)
	(stack.@= (newNode))	
	(retrun stack)
))
(fun (<<< stack) (
	(set node (stack.first))
	(if (node) 
		((stack.@= (node.next)) (return node.val))	
		((return node))
	)
))
(fun (>>> x stack) (
	(set node (newNode))
	(node.:= x)
	(node.-> (stack.first))
	(stack.@= node)
	(return stack)
))
(method.1 (first) (return this.1))
(method.1 (@= x) ((set this.1 x) (return this)))
(method.2 (next) (return this.1))
(method.2 (val) (return this.2))
(method.2 (-> x) ((set this.1 x) (return this)))
(method.2 (:= x) ((set this.2 x) (return this)))
(fun (newNode) (return (new 2))) 


\end{lstlisting}

\end{document}
