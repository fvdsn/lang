\documentclass[a4paper,fleqn,11pt]{article}

% Chargement d'extensions
\usepackage[latin1]{inputenc}
\usepackage[T1]{fontenc}
\usepackage[french]{babel}
\usepackage{listings}

% Modification des marges
\setlength{\oddsidemargin}{0.5cm}
\setlength{\textwidth}{15cm}
\setlength{\textheight}{23.5cm}
\setlength{\topmargin}{-2cm}
\setlength{\headheight}{1cm}

% Modification du style
\setlength{\parindent}{0pt}
\setlength{\parskip}{10pt}

% Style des listes
\lstset{frame=tblr,basicstyle=\sf\scriptsize,columns=fixed,aboveskip=5mm,belowskip=2mm,frameround=tttt,framesep=10pt,xleftmargin=10pt,tabsize=4}

% D�but du document
\begin{document}
\thispagestyle{empty}

% Num�ro du groupe
\hfill Groupe 6

% Nom du langage
{\Huge \sl \bfseries HAPPY-)}


% Exemple qui calcule la factorielle
\vspace{5mm}
\begin{lstlisting}
(fun (main) 
	(write (fact (read)))
)

(fun (fact x) (
	(if (= x 0) 
		(return 1)			
		(return (* x (fact (- x 1))))			
	)	
))
\end{lstlisting}

% Description
\vspace{1cm}\hspace{1cm}
\begin{minipage}{10cm}
\sf La syntaxe du HAPPY-) est simple et �pur�e � l'image du LISP dont notre langage est inspir�. Tout est une fonction (ou une m�thode) dans ce langage sauf le while et le if. le write renvoie
\end{minipage}


% Un autre petit programme au choix
\vspace{1cm}
\begin{lstlisting}
// Construit une liste d'entiers
main()
{
	last := null;
	read (x);
	while (x != -1)
	{
		node := new/2;
		node.1 = last;
		node.2 := x;
		last = node;
		read (x);
	}
	y := sum (last);
	write (y);
}

// Calcule la somme des entiers d'une liste
sum (list)
{
	s := 0;
	current := list;
	while (current != null)
	{
		s := s + current.1;
		current := current.2;
	}
	return (s);
}
\end{lstlisting}

\end{document}
